\hypertarget{course4}{}\chapter*{COURSE 4: The Interview}\label{course4}\addcontentsline{toc}{chapter}{COURSE 4: The Interview}\extramarks{}{COURSE 4: The Interview}
\noindent\makebox[\linewidth]{\rule{\linewidth}{0.4pt}}
\localtableofcontents
\noindent\makebox[\textwidth]{\rule{\linewidth}{0.4pt}}
\newpage

\pagebreak \section*{Module 1: Disclosure Review \& Disability Statement}\addcontentsline{toc}{section}{Module 1: Disclosure Review \& Disability Statement}\extramarks{Module 1}{COURSE 4: The Interview}
\noindent\makebox[\textwidth]{\rule{\linewidth}{0.4pt}} \etocsetnexttocdepth{4}
\localtableofcontents
\noindent\makebox[\textwidth]{\rule{\linewidth}{0.4pt}}
\leftskip=0.5cm
\pagebreak \subsection*{1-1\quad Lesson: Reviewing and practicing your disclosure strategy and disability statement}
\addcontentsline{toc}{subsection}{1-1\quad Lesson: Reviewing and practicing your disclosure strategy and disability statement}
Review of the information presented in Course 3 on disclosure and your disability statement is necessary at this point. Disclosure is an issue that persons with disabilities encounter throughout the employment process.

As you prepare for an interview you should be well versed on disclosure and have practiced your disability statement. Becoming successfully employed can depend on how well you explain and express appropriate information about your visual impairment, as well as how quickly you can make a potential employer comfortable with you and your blindness or low vision.

Take this time to review the information presented in the Course 3 lessons on disclosure and disability statements.

\pagebreak \section*{Module 2:	Company Research}\addcontentsline{toc}{section}{Module 2:	Company Research}\extramarks{Module 2}{COURSE 4: The Interview}
\noindent\makebox[\textwidth]{\rule{\linewidth}{0.4pt}} \etocsetnexttocdepth{4} \localtableofcontents
\noindent\makebox[\textwidth]{\rule{\linewidth}{0.4pt}}


\pagebreak \subsection*{2-1\quad Lesson: Taking your first step in interview preparation.}
\addcontentsline{toc}{subsection}{2-1\quad Lesson: Taking your first step in interview preparation.}
Before going to an interview with an employer, it's important to do some investigative work. Research can be the key to having a good interview and impressing an employer. Your standard interview preparation process should always include acquiring solid knowledge about the employer, the field, the market, competitors, and clients. This kind of research helps you gain knowledge about the company and determine what type of information to seek during the interview itself.

Most companies have websites that include background on the business and/or a staff directory, along with any positive media coverage the company has received. In addition to reviewing the company site, you should also search the web to find news or articles about the business. This kind of information can provide a glimpse into the company's current issues, past successes, and role in the community and industry.

Remember: Whenever you are using the Internet for research, it's important to determine if the information you've found is reliable and credible. Sites that consist mainly of user-generated or single-author content, such as Wikipedia, bulletin boards, or blogs, are often unreliable sources of information. A large part of the process of proper Internet research is understanding who the sources are and how much you can trust them.

Below are some preliminary questions to give you an idea of the kinds of information you should pursue in your pre-interview research. Make sure to note the questions you can't find the answers to so that you can ask them during the interview itself.

This information can be found by doing research on the Internet, reading articles about, or profiles of, the company from media outlets, and using your network of contacts to find people who have knowledge of your target company and/or industry.

\subsubsection*{Basic Business Facts}
\begin{itemize}[leftmargin=1.0cm]
	\item How large is the company?
	\item How many employees does the company have?
	\item Can you tell if employees remain with the company for a long time? Often you can find this information if the company publishes employee biographies on their site.
	\item Does the company have more than one location? Where is your position of interest located?
	\item What is the basic purpose of the business?
	\item What services does the business provide or what does it sell? Is the business for-profit or non-profit?
	\item How long has the business been in existence? Is it a new company? Old? Has it been owned by the same people the whole time or have there been shifts in ownership?
	\item What is the basic history of the company?
	\item Who are the people who currently run the company?
	\item Who runs the division you want to work in? Who works in the division you want to work in? How large is the division you want to work in?
	\item Where does your position fit into the company's structure?
	\item What do the published employee biographies sound like? What sorts of backgrounds do the employees have?
\end{itemize}

\subsubsection*{Market Research}
\begin{itemize}[leftmargin=1.0cm]
	\item Can you find published annual reports for the company? What do they tell you about the company's health and history?
	\item Where and how does the company make its money?
	\item Who buys what the company sells? Other businesses? Consumers? How does the company sell their products or services?
	\item Has the company been selling the same thing for a long time, or has it changed and adjusted to trends, advances, or demand?
	\item How much of a market share does the company have? Are they the major player for their products or services in your city? Your state? The country? Worldwide?
	\item How much has the company grown in the past year? Five years? Is the market expanding? Staying the same? Shrinking?
\end{itemize}
\subsubsection*{Competition}
\begin{itemize}[leftmargin=1.0cm]
	\item Who else is selling or doing the same thing?
	\item Who are the major competitors for the company?
	\item How do the competitors compare in terms of size, revenue, products/services offered, market share, history? Is the main competition a new start-up or a long-established company?
	\item Is there any difference between your target company and their competitors when it comes to consumers?
	\item Can you find published annual reports for the competition? What do they tell you about the competition's health and history?
	\item Workforce. Work Structure. and Schedule Does the business have different shifts?
	\item Does the business allow flextime to meet transportation or family needs? How long is a typical shift?
	\item Do the employees' shifts switch or rotate?
	\item Does the employer operate on a set schedule?
	\item Are there clues in the published material about what kind of worker does well at the company? From what you've found, do you have a sense that workers at the company are happy and satisfied?
\end{itemize}
\pagebreak \subsection*{2-2\quad Assignment: Company Research}
\addcontentsline{toc}{subsection}{2-2\quad Assignment: Company Research}
Create a list of the facts you already know about the target company. Then create a list of questions that you would like to find the answers to. Use the questions in this section as a guide; think of additional questions that are specific to the position, industry, company, or location.

Now use the Internet, the library, and your network to find the answers to as many of your questions as you can. Write up your research in a clear and easy to reference way. Remember: good research takes time and patience.
\pagebreak \subsection*{2-3\quad Example Assignment: Josie's Pre-Interview Company Research}
\addcontentsline{toc}{subsection}{2-3\quad Example Assignment: Josie's Pre-Interview Company Research}

\subsubsection*{Target Company}: Getty Images-Lexington

\subsubsection*{What I know}: One location of an international online stock photography and footage provider. Second largest stock imagery house. Main competitor: Corbis. Lexington location is data management only, sales and customer service are handled in New York, Los Angeles, and Chicago. They have another data management location somewhere in Florida and probably at least one or two locations in other countries.

\subsubsection*{Basic Business Facts}

\subsubsection*{How big?} Very big company. 13 locations total. 9,000 employees; 130 at Lexington. Revenues of \$28 million last year.

\subsubsection*{Purpose}: Getty is a for-profit company that sells royalty-free and rights-managed photography and footage to news organizations, companies and individuals.
History: They have been around since 1965. Their online business model was launched in 1999. Originally founded by the Getty family. Now a public company run by board-elected senior management.

\subsubsection*{Employee Information}: Based on the published employee bios, it looks like many people have been with Getty for over 5 years, a few have been there longer than 10.
For the data management employees, it looks like most of them have extremely solid technical and computer science backgrounds, along with a high degree of fluency in database design and management.

\subsubsection*{Position Information}: It looks like there are 10 Digital Asset Assistant positions. The next level up is Digital Asset Associates (10); then Digital Asset Managers (3); then Digital Asset Vice Director (1) and Director (1).

\subsubsection*{QUESTION}: How does the work get divided among the assistants? Does each assistant work with a single Associate?

\subsubsection*{Market Research}

According to the annual report I found on the Getty site, the company's performance between 2000-2007 was stellar. 2007-2009 was flat or declining. It looks like they are coming back from the recession like everyone else. In general, the financial picture looks strong at this point.

The market, particularly with royalty-free imagery, is expanding rapidly.
Getty sells all of their products online. They have an extensive retail and customer service team that supports the online outlet and a massive web design and data team that keeps everything moving and updated.

Getty has been selling stock imagery since its inception and has changed and adjusted to trends, advances, and demand by staying current with its artists and journalists and also keeping up with technology in terms of retail and delivery of their product.

The company has a 30\% domestic market share and is a major player worldwide. Their main US competitor is Corbis Imagery who holds 38\% percent.

\subsubsection*{Competition}

Corbis Imagery is really the only serious competitor. Corbis is known for a more corporate, less artistic product. Getty and Corbis are in direct and close competition when it comes to current event imagery sales.

Corbis is also an old company with a massive and successful online retail business structure. Corbis is in good financial shape and their business took the same hit during the recession.

\subsubsection*{Work Structure}

The data management division operates 24 hours a day in order to provide on the minute imagery for newspapers and online news outlets. It looks like their asset management teams run in three shifts: 7-3; 3-11; 11-7.

It looks like the employees are happy but also have a lot of work. The time pressure seems enormous and never ending.

\subsubsection*{QUESTION}: What structures are in place to provide support when a massive news event happens, or some other circumstance arises where deadlines are tight and the workload is huge?


\pagebreak \section*{Module 3:	Self-Description}\addcontentsline{toc}{section}{Module 3:	Self-Description}\extramarks{Module 3}{COURSE 4: The Interview}
\noindent\makebox[\textwidth]{\rule{\linewidth}{0.4pt}} \etocsetnexttocdepth{4} \localtableofcontents
\noindent\makebox[\textwidth]{\rule{\linewidth}{0.4pt}}


\pagebreak \subsection*{3-1\quad Lesson: Preparing an effective answer to a common and challenging question}
\addcontentsline{toc}{subsection}{3-1\quad Lesson: Preparing an effective answer to a common and challenging question}
In almost every interview, there will be an open-ended self-description opportunity, along the lines of: ``So, tell me about yourself.'' Often the interviewer will begin the interview with this type of question, and your answer will be one of the first things the interviewer hears from you. How you respond can set the tone of the interview, provide some direction for further questions from the interviewer, and establish your personality and attitude.

It's important to be prepared to answer this question confidently, clearly, and precisely, and to use your response to cover ground that you have thoughtfully considered. You should aim for an answer that runs between 1.5 and 2 minutes long. Because this is such a standard interview question, it's crucial to appear well- prepared to answer it-	a rambling, disorganized response, or an answer that is either too long or too short, can indicate that you did not prepare well for the interview.

There is a method to designing a proper and appropriate answer to this question. Below is a bulleted list of areas to cover, with some tips to keep you on the right track.
\begin{itemize}[leftmargin=1.0cm]
	\item Remember that your answer should be under 2 minutes in length.
	\item Mention one or two positive personal traits: enthusiastic, hardworking, diligent, organized, patient, etc.
	\item Talk briefly about where you grew up and your family. Example: ``I grew up in North Central New Jersey with my parents and two brothers.''
	\item Mention any accomplishments (e.g., Eagle Scout, student body president, athlete, etc.), but keep it short and to the point.
	\item Transition to work-related information or information that will demonstrate why you would be an asset to the business.
	\item Speak about any training or related experience that would be relevant to the position: degrees, courses, certifications, work experience, etc.
	\item If you volunteer for any organizations or charities, include this information after you've talked about job- relevant training and paid experience.
	\item Have a clear closing for your answer.
	\item Only elaborate if the interviewer asks you to clarify something you mentioned.
\end{itemize}
Interviewers value concise answers that have specific points. Many interviewers must make sure that interviewees meet the requirements listed for the position. Keep this in mind as you craft your answer.
\begin{itemize}[leftmargin=1.0cm]
	\item Use appropriate language and grammar
	\item Do not share irrelevant or negative information
	\item Be calm and composed
	\item Pace your answer, don't rush
\end{itemize}
The preparation you devote to this answer will serve you well in any interview situation.

\pagebreak \subsection*{3-2\quad Assignment: Self-Description and Disclosure Review}
\addcontentsline{toc}{subsection}{3-2\quad Assignment: Self-Description and Disclosure Review}
Write a short self-description that answers the question: ``So, tell me about yourself.''

Keep your answer between one-and-a-half and two minutes. Once you have written and polished your answer, memorize it and practice reciting it to a friend, family member, or teacher.

\subsubsection*{NOTE}:

Please remember to conduct a thorough review of your Disclosure Strategy and Disability Statement while you also create a list of the facts you already know about the target company. Then create a list of questions to which you would like to find answers. Use the questions in this section as a guide; think of additional questions that are specific to the position, industry, company, or location.

\pagebreak \section*{Module 4:	Addressing Employer Concerns}\addcontentsline{toc}{section}{Module 4:	Addressing Employer Concerns}\extramarks{Module 4}{COURSE 4: The Interview}
\noindent\makebox[\textwidth]{\rule{\linewidth}{0.4pt}} \etocsetnexttocdepth{4} \localtableofcontents
\noindent\makebox[\textwidth]{\rule{\linewidth}{0.4pt}}


\pagebreak \subsection*{4-1\quad Lesson: Addressing typical concerns for potential employers considering hiring a person with a disability}
\addcontentsline{toc}{subsection}{4-1\quad Lesson: Addressing typical concerns for potential employers considering hiring a person with a disability}

Unless they have done so in the past, employers may be unsure of hiring a person who is blind or low vision. Chances are that you will be the first person with a visual impairment your employer has interviewed or even met. It's natural to be nervous about something you don't understand or have no experience with. In the context of your professional relationships, your job is not to educate your potential employers, but to make them more comfortable working with you. By law, employers can't ask candidates about disabilities or impairments, but chances are that they will have concerns and questions. If you proactively address the areas that most employers have concerns about-and discuss your situation with tact, grace, and a positive spin- you increase your chances of getting hired.

There are three areas that should be considered when addressing an employer's concerns about visual impairment. Those areas are liability, accessibility, and transportation.

\subsubsection*{Liability}

Liability can be broken into two parts: 1) safety on the job, and 2) productivity.

\subsubsection*{Safety}

Businesses are always concerned about any employee getting hurt on the job or causing conditions that might lead to someone else having an accident. In today's world, employers must consider worker's compensation, liability insurance, and lawsuits. Typically, employers who are uneducated about disability will assume that if they should hire a person with a visual impairment, he or she will have a higher accident rate on the job. The truth is that persons with disabilities have no more accidents on the job than any other employee.

There will be employers who are not worried about such issues, but there will also be employers who are scared of you using the stairs. Employers who have safety concerns are not trying to be insensitive, they really just do not know the facts. Bringing up activities that you participate in such as sports, outdoor activities, or working out can help them get a more realistic perspective on safety issues.

\subsubsection*{Productivity}

Productivity concerns can be handled in many ways. Having good references and recommendations who are willing to express to an employer that you were able to complete past job duties can make a big difference.

Having a potential employer speak with a former colleague or supervisor is sometimes the most efficient way to communicate your abilities and potential. It's a good idea to talk openly about this with your references before you pass along their contact information to any potential employer.

Make sure that your references are comfortable vouching for your ability and make sure they know you are okay with them discussing your performance with a potential employer and answering any questions about your visual impairment in the workplace. If you do not have prior work experience, a club advisor, coach, volunteer supervisor, or someone besides friends and family who can vouch for your work abilities is a good substitute.

Some employers may believe that hiring a person with a visual impairment will be a liability because there will be a reduction in productivity. Maybe they think a person with a visual impairment would be a slower worker, or somehow wouldn't feel accountable for the work that he or she does.

The employer could think that a worker who is blind or low vision would constantly need assistance from other employees. Of course, this is not true, particularly with so many of today's jobs being computer-based and therefore more accessible to persons with visual impairments. Technology decreases the limitations in the workplace and can help to overcome most obstacles, but often potential employers might not understand this. You may be able to address this concern by giving examples from your past work experience, or by talking about the ways in which you use technology to help you be productive.

\subsubsection*{Accessibility}

Accessibility can mean all sorts of things to an employer. A common concern is how you will be able to acquire the information contained in written materials. Will they have to provide braille versions? What do they have to do to make a computer system accessible to you? Will you need special equipment?

If you have low vision then you can explain about your use of a CCTV, video magnifier, handheld magnifier, or screen magnifier. Screen reader users will want to explain how the software works in simple terms and can refer employers to websites if need be. Braille should not be an issue because you should be able to convert your own documents if that is needed. There may be a need for an embosser or refreshable braille display depending on your needs. You don't need to go into detail about how the technology works-again, you are not trying to educate the employer-you should talk about it in a way that demonstrates you can join an established business workflow easily and with little disruption to your coworkers' standard business practices.

\subsubsection*{Transportation}

Transportation is another area potential employers can have concerns about. Let them know that you regularly use the bus system, a car service, taxis, a private driver, trains, bikes, or good old foot power to get around. Mention how you got to the location on that day such as, ``Oh, I took the bus here today and had time to stop to get a bottle of water in the lobby. It's great that your building has those vending machines downstairs.'' You want to ease their fears and show them that your transportation is not their concern.

\subsubsection*{Resources for Employers}

An interview is a good opportunity to proactively ease these standard employer concerns. An interview can also be a good time to talk about the benefits of hiring a person with a disability-after all, persons with disabilities tend to be extremely reliable and loyal employees.

The For Employers sections of APH CareerConnect and NSITE Connect have information that employers may be interested in reviewing. When on a job interview, keep in mind that this is a resource you can recommend to them. It's a resource available to you as well-review the For Employers section to familiarize yourself with the facts and topics covered there.
\pagebreak \subsection*{4-2\quad Assignment: Addressing Employer Concerns}
\addcontentsline{toc}{subsection}{4-2\quad Assignment: Addressing Employer Concerns}

Write out what you would say to an employer to answer each of the concerns below.
\begin{itemize}[leftmargin=1.0cm]
	\item Liabilities
	\item Safety
	\item Productivity
	\item Accessibility
	\item Access to print materials (letters, forms, etc.)
	\item Messages between coworkers
	\item Transportation (getting to and from work)
\end{itemize}

\pagebreak \section*{Module 5:	Answering Interview Questions}\addcontentsline{toc}{section}{Module 5:	Answering Interview Questions}\extramarks{Module 5}{COURSE 4: The Interview}
\noindent\makebox[\textwidth]{\rule{\linewidth}{0.4pt}} \etocsetnexttocdepth{4} \localtableofcontents
\noindent\makebox[\textwidth]{\rule{\linewidth}{0.4pt}}


\pagebreak \subsection*{5-1\quad Lesson: Thinking about, and preparing answers to, common interview questions}
\addcontentsline{toc}{subsection}{5-1\quad Lesson: Thinking about, and preparing answers to, common interview questions}
Most interviews are structured around common questions designed to allow the employer to find out more about you and your potential to be a good fit with the organization. While speaking with you, the interviewer will most likely take notes on your answers. With good preparation for interview questions, you have the opportunity to portray yourself in the best light and to have clear and concise responses practiced and at the ready.

We've discussed preparing for the open-ended self-description question. In addition to this self-description category there are a few additional categories of common interview questions
\begin{itemize}[leftmargin=1.0cm]
	\item Strengths and weaknesses
	\item Future/commitment
	\item Work ethic/work personality
	\item Biggest accomplishment
	\item Ethics
	\item Psychology
	\item Fun/Leisure
	\item Your questions
\end{itemize}
\subsubsection*{Strengths and Weaknesses}:
\break ``What is your biggest strength?''
\break ``Name three weakness and tell me how you are working to improve on them.''

To prepare an answer to these types of questions, begin by referring back to section 1.5. Review the Skills and Abilities assignment. Think about which skills will benefit the organization and the position the most and think about how best to talk about these skills clearly and concisely.

The more challenging part of this question is talking about your weaknesses. It's best to start by phrasing positive attributes as weaknesses. Some interviewers will call you out on this and you'll need to have an alternate answer, but it's safest to begin with positives. How do you spin a positive as a weakness?

Here are a few examples:
\break ``I can become compulsive about performing well at work.''
\break ``Because work is something I am so passionate about, I tend to work too many hours.''
\break ``I have a hard time saying, 'no' to work assignments and I end up taking on a lot.''

\subsubsection*{When developing an answer for weaknesses, make sure not to re-use a quality or qualities you're using as a strength.}

You should also prepare a less-positive weakness so you can explain how you deal with or are working to improve.

Here are some examples:
\break ``I tend to overextend myself at times, but I'm getting better at achieving a good balance.''
\break ``My spelling is not the best, but I use spell check and an online dictionary to counteract this issue.''
\break ``I sometimes do not budget my time well, so now I use a personal planner to keep on top of things.''

\subsubsection*{Future/Commitment}:
\break ``Where do you see yourself in five (two, ten, fifteen) years?''

Companies want to hire people who are interested in making a long-term commitment and who want to grow with the company and participate in its success. Be imaginative when answering this question: what are your aspirations, ambitions, and vision for yourself at the company? They're not going to follow up with you in five years to see if you've accomplished what you state in your interview, so don't be shy!

One example of an answer:
\break ``In five years, I see myself in an upper-management position that allows me to have a wider influence on the company's growth and direction.''

\subsubsection*{Work Ethic/Work Personality}:
\break ``How would you describe your work personality? Can you give me examples from your prior positions?''
\break ``Why did you leave your last position?''
\break ``What did you like about your last job? What did you dislike about it?''

Employers ask these sorts of questions to try to get a sense of the kind of employee you'll make. The interviewer is looking for qualities that will be a good fit for the position and the company, and a personality that will fit in with the professional culture of their workplace.

If you do not have prior work experience, you should answer these questions by explaining how you have demonstrated work-appropriate skills through volunteering or participating organizations, clubs, school, and other activities.

Beware of over-sharing in your answer to questions about your prior positions. If you were fired, then you should be honest about it, but portray it as a learning experience that has made you a better employee.

\subsubsection*{If you resigned or moved on to a different position, here are some examples of short answers that don't give too much information:}
\break ``I left the organization because I felt underutilized.''
\break ``I felt it was time to move on to a better opportunity.''
\break ``I was offered a better opportunity.''
\break ``I went back to school.''
\break ``I relocated ...''

\subsubsection*{Ethics}: ``If you found out another employee was stealing or lying about their hours, what would you do?''

Some employers have had issues with employees taking advantage of the company or being dishonest while on the job, \textit{e.g.}, lying about the hours they work, skimming money from the register, stealing company supplies, using company resources (cars, credit cards) for their own needs, or observing coworkers doing these activities without reporting them.

While most applicants will not admit if they've done these things, employers will try to get a sense of your ethical sensibilities by talking about your standards for reporting coworkers.

An example of an answer:
\break ``I would report any employee I felt was behaving dishonestly in the workplace. I think trust between an employer and employee is very important and I'm not comfortable working in an environment where employees take advantage of an employer.''

\subsubsection*{Biggest accomplishment}:
\break ``Name the one accomplishment of which you are most proud so far in your career.''

Choose an accomplishment that shows your work ethic, determination, or skills related to the job. If you don't have a work history yet, choose an accomplishment such as completing training or getting a degree.

An example:
\break ``My biggest accomplishment so far is successfully completing my training as a chef. It took a lot of discipline and hard work, but I learned a tremendous amount and feel it has left me well-prepared for my chosen career.''

\subsubsection*{Psychology}: ``If you were a tree, what kind of tree would you be?'' ``If you were an animal, what animal would you be?''

Psychological questions can be hard to prepare for and may seem strange. They have become less common in the interview process because most employers don't know how to grade an answer to this type of question. If you do get this sort of question in an interview, take it seriously and keep in mind that the employer is looking for an answer that shows who you are or how you see yourself as a person or employee.

Some possible answers could be:
\break ``Palm, because they are flexible, yet strong.''
\break ``Teak, because it is considered one of the hardest trees and I can bear a lot of weight on my shoulders.''
\break ``Ebony, because it is considered one of the strongest and can produce extremely beautiful wood.''
\break ``I would be a Jack Russell Terrier because they are considered the most intelligent dogs.''
\break ``I would be a German Shepherd because they are loyal and obedient working dogs.''
\break ``I would be a work horse such as the Clydesdale because I mean business when I am at work and intend to get the job done.''

\subsubsection*{Fun/Leisure}:
\break ``What are your favorite hobbies? What do you do with your free time?''

These questions help employers get to know more about you. Choose a hobby or leisure activity that is appropriate and not controversial. Remember that this is a job interview, and you will be judged on your answer. Some safe areas are typically sports, music, literature, crafts, movies, theatre, hiking/camping, writing/arts, and philanthropic work. Philanthropic or volunteer work is always a thing good to mention, as it shows you are interested in helping other people.

\subsubsection*{Your questions}: ``Do you have any questions for me?''

Usually, interviewers ask this sort of question towards the end of the interview. If you are meeting with multiple people, each person may ask you this question. You should always have a list of questions prepared for the interview. Some may be answered as you work your way through the interview, but some will not.

\subsubsection*{Here are some sample questions}:
\break ``Is this a new position? If yes: Why did you feel the need to add it? If no: How long had the prior employee held the position? Why did he or she leave the position?''
\break ``What are the hours typically?''
\break ``What is the turnover rate like for this position?''
\break ``Why do you like working here?''
\break ``Describe the ideal employee for this position.''
\break ``Does this position have the opportunity to grow?''
\break ``What is the possibility of advancement within the business?''
\break ``Can I provide you with any more information to help you get a better idea about the quality of work that I would provide?''
\break ``Does the company offer benefits? What kind?''
\break ``What is the next step in the hiring process (only if they have not mentioned this prior)?''

It's extremely important to prepare for an interview by making sure you have good answers to the most common questions. Your goal is to make sure you are not caught off-guard in an interview and therefore likely to give a less-than-ideal answer, and also to be able to conduct the interview with less anxiety because you'll know you are well prepared.

\pagebreak \subsection*{5-2\quad Assignment: Answer common interview questions}
\addcontentsline{toc}{subsection}{5-2\quad Assignment: Answer common interview questions}
Write an answer for each question below and practice responding to these questions and any others you think an interviewer may ask. There are many internet resources that you can find that list common interview questions. Aim to answer each question in under two minutes. Once you feel prepared, ask a friend, family member, instructor, counselor, or teacher to ask you these questions and any others they think might come up in your interview. Practice answering-from memory-in a calm and collected manner.

\subsubsection*{Self-Description}
\break ``Tell me a little about yourself.''

\subsubsection*{Strengths \& Weaknesses}
\break ``What is your biggest strength?''
\break ``Tell me two of your weaknesses and how you compensate for them.''

\subsubsection*{Future/Commitment}
\break ``Where do you see yourself in five years?''
\break ``Where do you see yourself in ten years?''

\subsubsection*{Work Ethic / Work Personality}
\break ``What prior experience do you have that would relate to this position?''
\break ``What would you do if a big project was coming due the next day and it is not finished?''
\break ``How do you handle deadlines and meeting them? Give me an example.''
\break ``Can you give me an example of what you did when you had to deal with an extremely angry customer on the phone?''

\subsubsection*{Biggest Accomplishment}
\break ``What would you say is your biggest accomplishment and why?''
\break ``What is your biggest accomplishment and how did you accomplish this task?''

\subsubsection*{Ethics}
\break ``If you found out another employee was stealing or lying about their hours, what would you do?''
\break ``If a customer mistakenly gave you a \$20 bill instead of a single, how would you handle this matter?''

\subsubsection*{Psychology}
\break ``If you were a tree, what kind of tree would you be and why?''
\break ``If you could be any animal, what animal would you be and why?''

\subsubsection*{Fun/Leisure}
\break ``What is a hobby or leisure activity that you participate in?''
\break ``Do you participate in any charities or volunteer work? If so, what?''

\subsubsection*{Any Questions?}
\break ``Do you have any questions for me?''

\pagebreak \subsection*{5-3\quad Example Assignment: Josie's Answers to Common Interview Questions}
\addcontentsline{toc}{subsection}{5-3\quad Example Assignment: Josie's Answers to Common Interview Questions}
\subsubsection*{``Tell me a little about yourself.''}
\break I was born and raised in Southern California. My parents are both hard-working university professors and I have two younger brothers, both in college right now. Even when I was really young, I was always really interested in computer programming, math, and databases. I have this natural instinct to look for improvements to existing solutions, or to making something work more efficiently. For example, back in junior high school I put together a little database that organized the refreshment schedule for my volunteer service organization. You could make adjustments really easily and print out an updated schedule whenever you needed it. I think I apply the same drive to the professional challenges I have now, only the challenges are more complex than a refreshment schedule! All of my educational interests and independent training have gone into getting the knowledge and background I need in order to solve more and more complicated data- centered situations.

My fluency with Cold Fusion, SQL, and even Access helps me think of fast solutions and workarounds to data problems and I'm sort of obsessed about learning more and keeping my skills up to date. During my senior year, I completed three extra certifications for C++ just because I felt like I had some weak spots in my knowledge base. I'm also really invested in maintaining good data protocols and I love working with huge amounts of data and complicated databases. Last summer I interned at Apex Systems and helped them whip their inventory database into shape. I restructured and customized the system they were using, imported the data from the old structure, did a quality check, and then wrote some custom interfaces so that their inventory clerks wouldn't have to deal with a lot of technicalities. It was a huge improvement and I loved it. That's why I think I'd be such a good match here at Getty-it's the kind of data environment that I thrive in.

\subsubsection*{``What is your biggest strength?''}
\break My biggest strength is the passion and commitment I bring to the job. I love database management and asset management and I'm invested in performing at a high standard in this kind of work. I understand the consequences of mistakes in large database management, and I think that gives me a good perspective and intensity in what I do. Passion is also my main motivator. I'm passionate about constantly improving the work that I do, and helping others improve the work they're doing.

\subsubsection*{``Tell me two of your weaknesses and how you compensate for them.''}
\break I can have a tendency to get a little almost over-focused when I'm working on a problem. I think that this tendency can work to my advantage because it means I can really tackle difficult problems and I don't stop until I've arrived at a solution, but I recognize that in some cases it would be better to back off and take a wider view. I'm really working on taking an assessment in each work situation so that I can apply my tenacity and direction appropriately and effectively.

I love solving problems with computers, so my tendency is to always approach a challenge by applying a computer-based solution. Sometimes, though, a faster or more elegant solution is available in the ``analog'' world. I recently realized that I had complicated my mom's life by setting her up with a digital calendar. It's so much easier for her to just write something on a paper calendar, and for her it's clearly a more reliable way to keep track of her life.

\subsubsection*{``Where do you see yourself in five years?''}
\break In five years, I hope that I will be a Digital Asset Manager at Getty. I really admire this company and I can tell by the biographies you have on your site that you have a committed and talented workforce. I want to work for a company like yours and to make a long-term commitment. I think that I will learn very quickly and show myself to be a good leader and great worker.

\subsubsection*{``Where do you see yourself in ten years?''}
\break I would love to be Director of Digital Asset Management-or at least the heir-apparent to the position. I love this kind of work and I want to apply my skills to areas of broader and deeper influence.

\subsubsection*{``What prior experience do you have that would relate to this position?''}
\break All of my experience and education relates to this position. I've been a tech/math/computer nerd my whole life. It's just who I am. I've interned every summer at various local businesses helping them with computer issues, from setting up networks to overhauling inventory databases. I recently helped the library at my school troubleshoot a database problem they were having. You'll see on my resume that I have a lot of certifications-all of that was purely self-motivated. I didn't do it for school, I did it because I wanted to learn and then apply that learning to solutions.

\subsubsection*{``What would you do if a big project was coming due the next day and it is not finished?''}
\break If the situation were in my or my team's control, I would keep working until it's finished. That's actually my natural tendency anyway. I don't like leaving things unfinished and I hate missing deadlines. If the situation involved other constituencies, I'd work with them to come up with a plan and I would provide whatever work or service was needed to help finish the project. I'm very much a roll-up-my-sleeves sort of person-I'll Xerox, staple, make deliveries, whatever it takes-I don't mind going outside of my job description if it's going to help the team or company.

\subsubsection*{``How do you handle deadlines and meeting them? Give me an example.''}
\break As far as I'm concerned, deadlines are as much a part of a project's requirements as something like the data structure is. I take deadlines very seriously and while I know that every now and again something can come up that's beyond everyone's control, for the most part I feel that with proper planning and commitment, deadlines should be reachable. The most recent deadline example I can give you is that I had to turn in a final database design for my SQL programming certification and my little brother dropped his coke on my laptop and killed it. It was 9 at night with the project due the next day. I drove over to my friend Kate's house, who was the only other person I know who has SQL and she let me work in the living room until I was done. I had to start the project from about mid-way, but I managed to complete it by 2 am. It was a good database design, too!

\subsubsection*{``Can you give me an example of what you did when you had to deal with an extremely angry customer on the phone?''}
\break I haven't had to deal with angry customers, but I have had to deal with upset clients. Not upset with me, but just upset and frustrated by the fact that their computer situation isn't working. I've found the best way to handle it is to be a very proactive and calm problem solver, to get them confident that I'll fix it, and to get them out of the frustrating situation as soon as possible.

\subsubsection*{``What would you say is your biggest accomplishment and why?''}
\break My biggest accomplishment is probably the database overhaul I did for Apex. It's not the most complicated database I've worked on, but there were a lot of different people I had to talk to, a lot of different priorities to organize and address, and a lot of negotiation. I was really proud of the end product because I think it really helped them and made their work easier.

\subsubsection*{``If you found out another employee was stealing or lying about their hours, what would you do?''}
\break I would talk to my supervisor about it. I don't want to work in an environment where that's going on. It's just not my style.

\subsubsection*{``If a customer mistakenly gave you a \$20 bill instead of a single, how would you handle this matter?''}
\break I would tell them they gave me a \$20.

\subsubsection*{``If you were a tree, what kind of tree would you be and why?''}
\break I think I'd be an oak tree. To me, oak trees have a kind of quiet confidence. They're not showy, but they are impressive. Plus, they grow old well and have long lives!

\subsubsection*{``If you could be any animal, what animal would you be and why?''}
\break Well, I've always been a bit jealous of cats because they seem to not mind being lazy, but I think that's just not in my constitution. I think I'd be a sea tortoise. Again, the quiet confidence and they are very wise-seeming creatures. They're the kind of animal that makes everything around them get a little quieter and calmer, not because of slowness, but because of intelligent poise. They know what they need to attend to, and that's what they pay attention to. They don't get caught up in ``drama'' and I really like that.

\subsubsection*{``What is a hobby or leisure activity that you participate in?'' }
\break I like kayaking. It's really fun to be out on the water.

\subsubsection*{``Do you participate in any charities or volunteer work? If so, what?''}
\break I volunteer at a food pantry, and I am a relief caller for a peer-to-peer helpline.

\subsubsection*{``Do you have any questions for me?''}
\break Can you tell me more about the structure of the DAM division?
\break Has the current structure been in place long?
\break What are the company's priorities when it comes to digital asset management?
\break Would you describe the DAM division as working well right now?
\break What does the future look like for Getty and the stock imagery industry as a whole?
\break I'm just curious about how you would describe the difference between Getty and Corbis.


\pagebreak \section*{Module 6:	Preparing for the Interview}\addcontentsline{toc}{section}{Module 6:	Preparing for the Interview}\extramarks{Module 6}{COURSE 4: The Interview}
\noindent\makebox[\textwidth]{\rule{\linewidth}{0.4pt}} \etocsetnexttocdepth{4} \localtableofcontents
\noindent\makebox[\textwidth]{\rule{\linewidth}{0.4pt}}


\pagebreak \subsection*{6-1\quad Lesson: Getting ready to present yourself to the potential employer}
\addcontentsline{toc}{subsection}{6-1\quad Lesson: Getting ready to present yourself to the potential employer}
You've written out and practiced your answers to all of the questions you can think of. You've printed out clean copies of your resume and you've reviewed the published job posting so the published requirements are fresh in your mind. What's left to do?

\subsubsection*{Bringing Technology to the Interview}
\break You should be prepared to demonstrate the technology that you use for work or will need to bring to the interview. Create a checklist that lists all of the devices (high- and low-tech) you would use on the job. Next to each device listed, indicate whether you will bring the actual device or what you will bring to explain or provide as an example. It's ideal to bring the devices with you, but if you can't, bring pictures, video, website links or a description of each one. You could even create a short video or list of links to videos or sites that demonstrate your devices. The ultimate purpose is to demonstrate your methods of accessing information and completing work tasks.

We always have to keep in mind that most employers do not know how persons with visual impairments perform jobs. It's best to be able to show these methods or technologies quickly and efficiently. It's less likely that an interviewer will review these materials at a later point in time.

\subsubsection*{Presentation}

\subsubsection*{Dressing for the Interview}
\break Dressing appropriately can make the difference between getting a job and being eliminated as a candidate. Wear clothing that is clean (no stains), neat (no holes or tears), pressed (not wrinkled), and appropriately sized. Use a person you trust to view your clothing to see if it fits well. You should go to stores and try on clothing to find out what looks good and is comfortable. Trying on clothes is a necessity because clothing from different brands will fit differently even in the same sizes.

Different employers will have different dress codes. If you can ask someone what the dress code is before your interview, do so. If you can't, always err on the more formal/professional side.

Below are some general guidelines and tips for dressing appropriately for an interview. All organizations and jobs are different, but it's safest to dress conservatively, especially for a job interview. The best bet is to dress in a formal/professional manner when attending an interview. Review, try on, launder, iron, and hang your clothing a week before your interview so you have time to make adjustments or get things dry cleaned if needed.

\subsubsection*{Dress Tips Formal/Professional }
\begin{itemize}[leftmargin=1.0cm]
	\item Men
	      \begin{itemize}
		      \item Conservative suit (black, navy blue, or gray)
		      \item Sports coat, dress shirt, slacks, dress socks, dress shoes, tie, and belt (or suspenders)
		      \item Colors should match
		      \item Shirts should be a conservative solid color with a tie that matches
		      \item Shirt patterns should be subtle and minimal
		      \item Belt should be the same color as your shoes. If wearing a black or navy-blue suit, wear a black belt, black shoes, and black or navy-blue socks
		      \item A watch and/or one ring can be appropriate if formal If you have a talking watch, the alarm should be silenced; talking watches can be a distraction and should be used cautiously
		      \item Dress shoes should be polished and in good condition
	      \end{itemize}
	      Formal clothing does not have to be expensive or a top brand: look for sales or shop at a local thrift store. Know your sizes and try things on both before purchasing and prior to an interview-clothing that fits properly is important to presenting a professional appearance. A tuxedo is not appropriate for an interview. Men should always wear a white undershirt beneath their dress shirt to present a conservative appearance and prevent sweating through the shirt. Undergarments should not be visible, and clothing should not be transparent, nor form fitting.
	\item Women:
	      \begin{itemize}
		      \item Dress suit/pant suit
		      \item Jacket with slacks and an appropriate blouse
		      \item Jacket with a knee-length or longer skirt
		      \item Jewelry should be minimal and subtle: small earrings (if any), one necklace
		      \item Clothing should be conservative and fit properly
		      \item Formal clothing does not have to be expensive or a top brand: look for sales or shop at a local thrift store
		      \item Neckline should be conservative and not low. (Very little skin should be showing)
		      \item Shoes should be a dark color (black, brown, navy), closed toe, with a low or flat heel

		            If stockings are worn, they should be a neutral shade or one that matches your skin tone. Undergarments should not be visible, and clothing should not be transparent, nor form fitting. Handbags should be well-kept, moderate in size, neat in appearance, and devoid of distracting ornamentation.
	      \end{itemize}\end{itemize}
\subsubsection*{Business Casual (varies from business to business) }
\begin{itemize}[leftmargin=1.0cm]
	\item Men
	      \begin{itemize}
		      \item Dress shirt (button down shirt that is striped or a solid color) and slacks (Docker/khaki type pants), socks, belt, and dress shoes
		      \item Some businesses will require a tie
		      \item Certain businesses may allow a polo shirt as part of business casual instead of a dress shirt (if you're unsure, stay conservative)
	      \end{itemize}
	\item Women
	      \begin{itemize}
		      \item Conservative blouse or shirt, knee-length or longer skirt, dress of an appropriate length and neckline
		      \item Slacks can be substituted for a skirt/blouse or dress
		      \item Pantyhose/stockings might be required or recommended, depending on the company culture or location
		      \item Minimal jewelry
	      \end{itemize}\end{itemize}
\subsubsection*{Casual}

\subsubsection*{For Men and Women:}
\break For most interviews you should never dress any more casually than the business casual guidelines above. You may adjust your wardrobe as appropriate after you've been hired. If dressing to do a more labor-intensive job, ask what is suggested to be worn to the interview. You should stay away from any inappropriate or very casual clothing even if you are interviewing for a more labor-intensive job.

\subsubsection*{Regional/Cultural/Organizational Differences}
\break Some regions of the country and world have different professional dress conventions. It's important to respect the values of the organization and culture that you are applying to work within. Some regions are more casual about their dress because of the climate. For example, pantyhose would less likely be worn in Miami, Florida or Honolulu, Hawaii. Businesses in a region of the country that is known to have a traditional culture may be more conservative about dress codes. All of this is important to research and understand prior to an interview. Many businesses have written dress codes for employees (and interviewees) to follow.

\pagebreak \subsection*{6-1\quad Assignment: Getting ready}
\addcontentsline{toc}{subsection}{6-1\quad Assignment: Getting ready}
Create a checklist of materials and technology that you will need to demonstrate, describe, or relay resources for at your interview.

Make sure you have an appropriate interview outfit, clean and ready to wear. If you don't, make a list of the items you need to find and go shopping.

\pagebreak \section*{Module 7:	Interview  Coaching and Role Play}\addcontentsline{toc}{section}{Module 7:	Interview  Coaching and Role Play}\extramarks{Module 7}{COURSE 4: The Interview}
\noindent\makebox[\textwidth]{\rule{\linewidth}{0.4pt}} \etocsetnexttocdepth{4} \localtableofcontents
\noindent\makebox[\textwidth]{\rule{\linewidth}{0.4pt}}


\pagebreak \subsection*{7-1\quad Lesson: Preparing for the real thing}
\addcontentsline{toc}{subsection}{7-1\quad Lesson: Preparing for the real thing}
When preparing for an interview, it's not enough to know your material and understand the skills involved in a successful interview. You need to be able to actively apply these skills in an interview setting, where you'll need to respond to unexpected questions and situations. There is no ``pause'' or ``stop'' button in a live interview, nor are there ``do overs.'' When it comes to interviewing for a job, you have only one opportunity to make a good impression and present your strengths. Practicing your interview skills through role-playing and mock interviews is important for preparation.

\subsubsection*{Role-Playing an Interview}
\break Role-playing an interview is a great way to practice your interview skills. Find a friend, family member, teacher, rehabilitation professional, or other person you trust to help you-someone who has experience with interviews, and who will be comfortable giving you constructive feedback that identifies your strengths and weaknesses. Once you've found someone willing to help, prepare information on a job that you would be interested in getting and provide it to him or her, so that your helper can accurately represent the kind of interview you might encounter. Ask your helper to come up with a range of interview questions, from the most basic to the most challenging.

When you conduct the role-playing exercise, have your helper start off as the interviewer. Try to put all of the skills you've learned so far into practice when you answer questions, and don't forget to pay attention to how you are presenting yourself. If you have trouble answering a question, you and your helper can talk about improvements, try out different answers, and hone your responses until you feel comfortable. If you're stuck, switch roles and see how your helper would answer the question, then try it yourself. It may be helpful to record these role-plays to help you identify what you can improve on. Role-playing situations from an interview can help you feel more comfortable with difficult questions and can help you prepare to address a variety of issues that may arise during the interview.

\subsubsection*{Mock Interviews}
\break A mock interview is a more formal role-play with someone who will be nonpartisan and more critical of you as an interviewee. Preferably, your mock interviewer will be someone you do not interact with often. Use your personal network, friends, and family to find a business professional who would be willing to do a realistic interview with you. It would be preferable if the person has experience with interviews, either as an interviewee or interviewer. Provide your mock interviewer with as much information as possible about the job you're interested in, and make sure that you answer any questions they may have about the exercise before beginning the mock interview. Your mock interviewer should understand that the goal is to make the exercise as realistic as possible, and you should explain that you are looking for feedback on any and all aspects of your interview skills, from appearance and first impression down to how you answer specific questions.

The mock interview should be a full run through with no stopping. The more authentic the experience, the more valuable it will be. Dress appropriately, arrive on time, and conduct yourself from start to finish as if you were in a real interview.

Once the interview is over, it's important to obtain your mock interviewer's feedback, whether directly or through another person. It also could be helpful to get written notes or points to improve on, as well as areas that you excelled in. It's extremely important not to take any notes for improvement personally, nor to get upset if the mock interviewer has identified an area of your interviewing skills that needs work. The information provided by your mock interviewer is valuable and, if you pay attention to it, can help you grow as an interviewee.

\subsubsection*{Tips for Role Plays and Mock Interviews}
\begin{itemize}[leftmargin=1.0cm]
	\item Remember to greet the interviewer and thank him or her for the opportunity to meet.
	\item Sell yourself.
	\item Listen to the interviewer and answer all of his or her questions.
	\item Include necessary information and information that will represent you appropriately.
	\item Be enthusiastic.
	\item Be prepared to ask at least one or two questions about the job or business.
	\item Allow the interviewer time to speak; do not monopolize the conversation because the interviewer may have specific questions for you.
	\item Provide the interviewer with a resume at the beginning of the interview.
	\item Thank the interviewer for taking the time to meet with you.
	\item Note the names of all the persons that you meet with for thank you messages and other purposes.
	\item Use appropriate eye contact (do not stare into their eyes, try to aim for right above their eyes when you look at the interviewer).
	\item Shake the interviewer's hand in the beginning of the interview and at the end.
	\item Prepare well.
\end{itemize}
Often successful interviewees have specific points about themselves, their training, or job experience that they want to make sure to deliver to the interviewer. Identify the points you want to make in your interviews and practice talking about them in your role-playing and mock-interview exercises.

If you have to turn in documents or information to the interviewer, make sure these are in good condition (no interviewer likes crumpled pieces of paper).

Refer back to information on appropriate dress and hygiene.

It's important to note that the business world is small, and businesses have relations with other businesses- take mock interviews seriously and treat everyone professionally and respectfully. You might make a good impression on someone who can help you with your next steps or recommend you for a job. Alternatively, if you do not behave professionally or respect the mock interviewer's time and efforts on your behalf, you might damage your reputation within his or her network.

\pagebreak \subsection*{7-2\quad Assignment}
\addcontentsline{toc}{subsection}{7-2\quad Assignment}
This lesson prepared you for role plays and mock interviews. Utilize the Example Scoring Sheet to help you get accurate feedback from the persons you select to help you practice your interviewing skills. Answer the following questions to help yourself reflect on your experiences post role play and mock interview.

What were the positives? (What did you do well?)

What were the negatives? (What do you need to improve on?)

\pagebreak \subsection*{7-3\quad Example Scoring Sheet}
\addcontentsline{toc}{subsection}{7-3\quad Example Scoring Sheet}
\subsubsection*{Mock Interview Scoring Sheet}
\break This is an interview scoring sheet developed to help the interviewer review the interviewee's performance. The feedback should be given in an honest and constructive manner in order to help the interviewee identify strengths and weaknesses. Rate the interviewee in the following areas using the scale of 1 to 5 (1-Needs More Improvement, 2-Needs Improvement, 3-Satisfactory, 4- Distinguished, \& 5-Excellent).
\pagebreak

Interviewee Name: \hrulefill
\begin{enumerate}[leftmargin=0.5cm]
	\item Was the interviewee dressed appropriately?
	      \break Score: 1 2 3 4 5
	      \break Comments:

	\item Did the interviewee act in an appropriate manner for an interview?
	      \break Score: 1 2 3 4 5
	      \break Comments:

	\item Did the interviewee seem prepared for the interview?
	      \break Score: 1 2 3 4 5
	      \break Comments:

	\item Was the interviewee punctual? (Answer if appropriate)
	      \break Score: 1 2 3 4 5 Not Applicable \break Comments:

	\item Did the interviewee answer the questions appropriately?
	      \break Score: 1 2 3 4 5
	      \break Comments:

	\item What do you think the interviewee did well during the interview?
	      \break Comments:

	\item What do you think the interviewee could improve upon?
	      \break Comments:

	\item Overall Score: Did the student interview well?
	      \break Score: 1 2 3 4 5
	      \break Comments:
\end{enumerate}

\pagebreak \section*{Module 8:	Thank You Letters and Emails}\addcontentsline{toc}{section}{Module 8:	Thank You Letters and Emails}\extramarks{Module 8}{COURSE 4: The Interview}
\noindent\makebox[\textwidth]{\rule{\linewidth}{0.4pt}} \etocsetnexttocdepth{4} \localtableofcontents
\noindent\makebox[\textwidth]{\rule{\linewidth}{0.4pt}}


\pagebreak \subsection*{8-1\quad Lesson: Paying attention to post-interview etiquette}
\addcontentsline{toc}{subsection}{8-1\quad Lesson: Paying attention to post-interview etiquette}
You might think you can sit back and relax because you’ve completed your interview. Actually, it's time to get back to work! You should be writing thank-you letters or emails to all of the people you met with during your interview and to all of the people who provided you with leads or information on the position, business, and industry. This is why it's so important to keep information on everyone you met with and everyone who has helped you along in the process.

Sending thank-you letters or emails to interviewers is a social expectation when interviewing for a job. It's also another way you can stand apart from the rest of the candidates. Thank-you notes give you the chance to demonstrate proper etiquette, writing skills, and follow-through, and also show that you are aware of, and grateful for, the time and energy people have spent helping you and considering you for a position.

Write and send these thank you letters or emails promptly (within 48 hours of the interview) and compose them in a professional format. The recipients are not your friends and one of them may hold the key to you getting the job, so be formal, polite, and respectful. Send a message to each person you interacted with during your interview, including assistants and secretaries. You never know who has influence in the hiring process.

\subsubsection*{Here are some tips for writing thank-you letters or emails}:

Write the thank-you message in a word processing program so you can use spelling and grammar checks.

Letters should be composed in formal business letter format (you may want to review the 3.7 lesson on cover letter writing).

For emails, the letters should be both attached as a document and pasted into the body of the email. Here is an outline for guidance:

For the structure of the letter, review the lesson on cover letters.

\subsubsection*{Opening paragraph}: Thank them for their time and express your appreciation for the interview.

\subsubsection*{Second paragraph}: Offer a final sell of why you would be a great fit for the position. Keep this brief and respectful.

\subsubsection*{Closing paragraph}: Thank them again and say that you look forward to hearing from them in the near future.

Remember to thank all of the people who provided assistance to you in this process. This can be tedious, but this effort will pay off as you demonstrate professionalism and respect. Get started on those thank you letters or emails!

\pagebreak \subsection*{8-2\quad Assignment: Sample Thank-You Letter}
\addcontentsline{toc}{subsection}{8-2\quad Assignment: Sample Thank-You Letter}
Write thank-you letters or emails. Submit one letter to your portfolio to keep on file for your future reference. You can write your letter in a word processing program and paste it into your assignment. Get thanking!

\pagebreak \subsection*{8-3\quad Example Assignment: Josie's Sample Thank-You Letters}
\addcontentsline{toc}{subsection}{8-3\quad Example Assignment: Josie's Sample Thank-You Letters}
Josie Armentrout
\break 123 Main Street, \#34
\break Lexington, KY 10000
\break 999-444-1111
\break jarmentrout@gmail.com
\break May 13, 2016

Ms. Marion Fishman,
\break Store Manager
\break John's Supermarket
\break 123 Corporate Loop
\break Lexington, KY 10000

Dear Ms. Fishman:

Thank you very much for meeting with me on Tuesday to discuss the opening for the bagger position at your store. I appreciate the opportunity to learn more about the standard of excellence you hold your employees to at John's Supermarket. I am very excited about the possibility of joining your team.

You mentioned you need a reliable employee who works well with others. My experience as captain of my high school's Goal Ball team has helped me to develop the attributes you are looking for in an employee. I would go the extra mile for the store and work when you need additional assistance, such as when employees are out or during peak shopping times during the Holidays. Having grown up in our local community, I am familiar with many of the customers who shop in your store and would contribute to providing a high level of customer service with my friendly and outgoing personality.

Again, thank you for considering me for the open position on your crew. I would love to spend my afternoons after high school and weekends working and learning in your store. Please feel free to call me if you have any questions. Thank you again for your time and consideration. I look forward to hearing from you soon.

Sincerely,

Josie Armentrout


\pagebreak \section*{Module 9:	Following Up After an Interview}\addcontentsline{toc}{section}{Module 9:	Following Up After an Interview}\extramarks{Module 9}{COURSE 4: The Interview}
\noindent\makebox[\textwidth]{\rule{\linewidth}{0.4pt}} \etocsetnexttocdepth{4} \localtableofcontents
\noindent\makebox[\textwidth]{\rule{\linewidth}{0.4pt}}


\pagebreak \subsection*{9-1\quad Lesson: Responding strategically to good and bad news after the interview}
\addcontentsline{toc}{subsection}{9-1\quad Lesson: Responding strategically to good and bad news after the interview}
Now that you've completed and sent thank-you letters, it's time to follow up on your interview. Following up with an employer demonstrates you are interested in working for them and eager to get started. Not all employers contact you to let you know what decision they made, or where they are in the process, so it's important that you be proactive in keeping in the communications loop.

There are standards to follow when following up after an interview. Unless advised otherwise during the interview, it's typical to wait three days after the end of interviews to contact the employer. If you have been communicating mostly by email, it's acceptable to follow up via email-though typically businesses will share more detailed information over the phone.

At a larger organization, contact the personnel or human resources department. At a smaller organization, you should reach out to the person who has been your main point of contact. Be polite and gracious when speaking to anyone at the business. If they don't have an answer for you when you call, yet say that they will call you back, ask for a timeframe to expect a response. If you don't hear from them within the timeframe, follow up again. It's important to appear eager and enthusiastic, but not overly aggressive or demanding.

If you did not get the job, don't cry or share your emotions when you get the decision. Thank them for giving you the opportunity to interview. Remember, not everyone gets the opportunity to interview for a position and this shows that they valued you as an applicant. You never know if they will have any future openings that would fit your qualifications, so you want to make sure to maintain a professional demeanor. State your continued interest in the company.

It's appropriate to ask if they keep applications and resumes on file for future openings.

If you get the job, try your best to be calm and reserved. Express your gratitude. You can ask when they would like you to start. This can be negotiable in some cases. If you are presently employed, you can tell them that you would like to give your current employer the courtesy of two weeks (or whatever your company's standard notice period is), if possible.

If the employer wants you to start sooner, then you may have to leave your current position earlier. Jobs can be hard to find, and most employers will understand if you explain that you tried to get them more transition time.

You do not want to burn bridges if you are leaving another job. You never know if you will end up working with that employer again or interacting with them in your new position.

Leaving a job on good terms can provide a good reference in the future.

If you have prior commitments scheduled such as a vacation, this will be something you have to discuss. Some businesses will be okay with you taking that vacation, but many will not. Sometimes you will have to choose your job over personal life.

Confirm your pay or wage. How many hours does this business consider full time employment? Will they be providing any type of benefits? If you are receiving benefits, what kind of deduction from your pay will you have to contribute?

These are just a few points to consider when calling to follow up with a business that you have interviewed with. It's suggested that you practice what you will say and even make notes to make sure that you cover all bases.

\pagebreak \subsection*{9-2\quad Assignment: Your Interview Follow-Up Plan}
\addcontentsline{toc}{subsection}{9-2\quad Assignment: Your Interview Follow-Up Plan}
Write a detailed script of what you will say when following up after an interview. Write out what you will say and the questions you will ask if you did not get the job. Write out what you will say and the questions you will ask if you did get the job.

Remember to be polite and gracious no matter what the decision is.

\pagebreak \subsection*{9-3\quad Example Assignment: Josie's Interview Follow-Up Plan}
\addcontentsline{toc}{subsection}{9-3\quad Example Assignment: Josie's Interview Follow-Up Plan}
\subsubsection*{Script:}
\break Hello, my name is Josie Armentrout. I came in to interview with Ms. Fishman for the Digital Assets Management Assistant position on May 1 and I wanted to check on the status of the hiring process. Ms. Fishman told me that interviews were scheduled to be completed on May 22. Do you know if a decision has been made yet?

\subsubsection*{If I got the job:}
\break That's great! I'm thrilled.
\break Do you have a moment to answer a few questions? Or is there someone who I can speak to about things like benefits and start date?
\break I'd like to give my current employer three weeks, which is their standard request for notice. Would that schedule work for you?
\break Can you tell me what benefits are available to me? Is there an orientation for new employees?
\break Can I confirm the salary for the position?
\break Is there anything I need to do before my first day?

\subsubsection*{If I didn't get the job:}
\break Well, I am disappointed, but I really do appreciate the opportunity to interview. I really do want to work for Getty Images. Does the HR department hold resumes in case an appropriate opening comes up?
\break Thanks again for your time.

