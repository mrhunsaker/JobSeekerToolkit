\hypertarget{course3}{}\chapter*{COURSE 3: Finding Employment}\label{course3}\addcontentsline{toc}{chapter}{COURSE 3: Finding Employment}
\extramarks{ }{COURSE 3: Finding Employment}
\noindent\makebox[\linewidth]{\rule{\linewidth}{0.4pt}}
\localtableofcontents 
\noindent\makebox[\textwidth]{\rule{\linewidth}{0.4pt}} 
\newpage

\pagebreak \section*{Module 1: Disclosure}\addcontentsline{toc}{section}{Module 1: Disclosure}\extramarks{Module 1}{COURSE 3: Finding Employment}
\noindent\makebox[\textwidth]{\rule{\linewidth}{0.4pt}} \etocsetnexttocdepth{4}
\localtableofcontents 
\noindent\makebox[\textwidth]{\rule{\linewidth}{0.4pt}} 


\pagebreak \subsection*{1-1\quad Lesson: Disclosure}
\addcontentsline{toc}{subsection}{1-1\quad Lesson: Disclosure}
Deciding when to disclose your disability to a potential or current employer is one of the most substantial issues persons with visual impairments and disabilities encounter during the employment process. It's also one of the most frequently debated issues. If you ask three people who have disabilities about disclosure, you will get three distinctly different points of view.

Disclosure decisions might be easier for a person with a more obvious physical disability, but for people with low vision, or less apparent disabilities, disclosure can present a challenge. It's important to put careful thought into disclosure so that you understand your options and their potential consequences, both positive and negative. This section will guide you through the process of thinking about when and how to disclose your disability.

Ultimately, it's your decision to disclose or not, though it's important to remember that you're not covered under the Americans with Disabilities Act (ADA) until you disclose your disability to the employer. There is not one right answer to the disclosure issue; each situation and employer is different.

\subsubsection*{Disclosure Scenarios: Pros and Cons}
Below you'll find a discussion of stages in the employment process when you might consider disclosure, along with a discussion of some possible reactions and consequences. These are typical situations that you are likely to encounter during your job search.
Disclosure prior to the interview, by email or phone call 

\textbf{Possible Positives}:

The employer knows upfront and will not be caught off guard when you arrive at the interview.

The employer feels that you are being honest. The employer will have time to think about how a person with this type of disability would fulfill the job duties.
The employer may interpret your comfort with disclosing your disability at this early stage as a sign that you are confident you can do the job.

If the employer has had positive experiences in the past with persons with disabilities, they will be excited by the opportunity to hire someone with a disability.
If the employer has been given an initiative to hire competent persons with disabilities, they will recognize that your employment could be a great opportunity.

\textbf{Possible Negatives}:

The employer may be scared or intimidated and decide to ignore your resume or application due to misconceptions about people who are blind or low vision.
The employer may have never met a person with a visual impairment and thus be unsure or uncomfortable about interviewing you.

The employer may not intend to hire you because of the disability but will interview you because they are worried that you might accuse them of discrimination.
The employer may think they will not be able to afford the accommodations required to hire an employee with a disability and therefore may not interview you.
 
The employer may believe people who are blind or low vision always have multiple disabilities and that you will not have the intelligence to perform the job and the required duties.

\textbf{Disclosure prior to an interview}

By meeting the employer in person (going into the employer's office or place of business to retrieve or drop off an application) you may disclose that you have a disability.

\textbf{Possible Positives}:

Going in person demonstrates your ability to arrive at the employer's office and act professionally.

By entering with confidence and being dressed appropriately, you can make a good impression and your disability will not be an issue.

Whomever you meet will have the chance to interact with you and ask questions about accomplishing the job duties. This is an opportunity to sell yourself as a potentially valuable and capable employee.

In-person interaction will dispel any biases or misconceptions about your intelligence.

An in-person meeting is a good opportunity to educate a potential employer and promote yourself as a quality applicant in the process.

By disclosing at this stage, the employer can ask about possible accommodations that you would need to fulfill the duties of the job or to take an entrance test. Again, this is an opportunity to ease their worries by explaining what you need. You may have the opportunity to demonstrate a piece of technology that you have with you at that time.

\textbf{Possible Negatives}:

If the employer has a bias or holds prejudice against persons with disabilities, now that they know you are disabled, they might block you from getting an interview, even though your resume and supporting documents demonstrate that you are an appropriate candidate for the position.

The employer or staff might assume you will not be able to perform a job that sighted people typically perform. If they were to hire you, they think it would mean other employees would have to perform some or most of your responsibilities.

Some employers may think your disability disclosure is a scam to set them up for an Americans with Disabilities Act (ADA) non-compliance lawsuit.

\textbf{Disclosure during the interview, in person, or on the phone} 

\textbf{Possible Positives}:

Open-minded employers will be interested in how you see your abilities fitting with the job duties.

Some employers may be comfortable hiring persons with disabilities, but still have concerns. By disclosing at the interview stage, you have the opportunity to address their concerns and demonstrate that you are a competent, qualified candidate who would be a superb addition to their company.

Some employers may need a little education on your visual impairment.

Disclosing at the interview stage will give you the opportunity to inform them about the reality of your disability. When you choose to go this route, it's important to talk about your situation simply, honestly, and comfortably without making the interviewer feel dumb or awkward.

Remember: employers are not supposed to ask about disabilities; if you choose to disclose, they will want relevant information. It's better to proactively and fully address their concerns.

Some employers may have other quality employees who have disabilities and will be excited to see how you could do the job effectively.

Waiting to disclose until the interview means that a possible biased employer doesn't have the opportunity to block you from interviewing. You have a better chance of being judged fairly and also have the chance to perform well during the interview.

\textbf{Possible Negatives}:

The employer might feel that waiting to disclose until the interview is a dishonest way to represent yourself. They might not hire you because they feel you aren't trustworthy.
The employer or staff might assume you will not be able to perform a job that sighted people typically perform. If they were to hire you, they think it would mean other employees would have to perform some or most of your responsibilities.

The employer may feel uncomfortable or embarrassed because they didn't expect a person with a visual impairment, and they have a lack of experience with persons with disabilities.

The employer may have had bad experiences with persons with disabilities in the past or have heard stories of bad experiences. Disclosing during the interview gives them the opportunity to apply these negative feelings to you and your application/interview before they've made a hiring decision.

The employer may see your disability as a legal liability and won't want to run the risk of hiring you because they think you have a higher chance of getting injured, or that you would accuse them of discrimination.

\textbf{Disclosure right after you are hired and when already on the job}

\textbf{Possible Positives}:

You are hired without having to negotiate biases or run the risk of getting excluded from consideration because of your disability.

\textbf{Possible Negatives}:

Waiting to disclose until after you're on the job might make your employer feel they can't trust you. They may work to accommodate your disability, but you may have damaged a professional relationship.

Because they don't want to accommodate your disability, or because they are biased, your employer may find a non-related reason to fire you.
Your employer feels they have been forced into something without complete knowledge of the consequences. Your employer may interpret your delay in disclosure as a lack of confidence in your abilities.

Your employer may think you purposely waited to disclose so that you can pursue a lawsuit. This may damage your professional relationship and reputation.
Disclosure only when it becomes an issue on the job 

\textbf{Possible Positives}:

You are hired without having to negotiate biases or run the risk of getting excluded from consideration because of your disability.

You are able to prove your worth on the job.
 
\textbf{Possible Negatives}:

You have to hide something that is a part of who you are.

Your employer may notice that you struggle with some tasks and begin to think that you have a cognitive issue.

Your employer may become unhappy with your performance and begin documenting these issues. You may eventually disclose your disability to your boss, but at that point the decision to fire you may have already been made.

Because the employer was not aware of your disability at the time they made the decision to fire you, you are not covered under the ADA.

Because you disclose late in the game, the employer feels you are simply making excuses for work performance issues.

Because you wait so long to disclose, the employer feels that you have misrepresented yourself and not been honest with the staff.

Your employer may feel you are not comfortable with having a disability and this may make them uncomfortable around you.

\pagebreak \subsection*{1-2\quad Assignment: Disclosure}
\addcontentsline{toc}{subsection}{1-2\quad Assignment: Disclosure}
In this section, several disclosure scenarios were discussed, along with their possible positive and negative consequences. What is your reaction to each of these scenarios and their consequences? Why?
\begin{itemize}[leftmargin=*]
\item Disclosure prior to the interview, by correspondence (a letter or cover letter) or phone call
\item Prior to an interview, by meeting the employer in person (going into the employer's office or place of business to get or drop off an application)
\item During the interview, in person or on the phone
\item Disclosure right after you are hired and when already on the job
\item Disclosure only when it becomes an issue on the job
\end{itemize}
\pagebreak \subsection*{1-3\quad Example Assignment: Joe's Disclosure}
\addcontentsline{toc}{subsection}{1-3\quad Example Assignment: Joe's Disclosure}
I will be an asset to your company because I bring a good work ethic, reliability, loyalty, collaboration, and high standards to every job I do. I'm an excellent team player and have always been able to accomplish work tasks on my own.

My right field of vision is perfectly fine, but I've lost my left field of vision. This means I have fully functional, but partial, sight in both of my eyes. In order to adapt to this condition, I have to scan my surroundings in order to see everything. I'm able to read, type, and interpret visual information with no problem. As you can see, I get around easily and well in the physical world. I use the bus system to get around the city. I took a bus here today and the route is very straightforward. When the buses aren't running, I have alternative, reliable transportation.
 
\pagebreak \section*{Module 2: The Disability Statement}\addcontentsline{toc}{section}{Module 2: The Disability Statement}\extramarks{Module 2}{COURSE 3: Finding Employment}
\noindent\makebox[\textwidth]{\rule{\linewidth}{0.4pt}} \etocsetnexttocdepth{4} \localtableofcontents 
\noindent\makebox[\textwidth]{\rule{\linewidth}{0.4pt}} 


\pagebreak \subsection*{2-1\quad Lesson: Preparing for how to talk about your disability with a potential employer}
\addcontentsline{toc}{subsection}{2-1\quad Lesson: Preparing for how to talk about your disability with a potential employer}
At some point in the employment process, you may want to describe your disability and how it affects you in life and in the workplace. Doing so will allow you to be protected by the ADA and will also give you the opportunity to make your potential employer more comfortable and informed about what you can bring to the company. Whenever you decide to disclose your disability, it's important to think about how to talk about it in a professional, honest, and non-threatening manner.

As a person with a disability, it's important to express what your disability is and how it affects you. It's important to not use too many technical terms and to keep your explanation practical. Make sure to be clear about what you can see and explain how you accommodate limitations. Mentioning how you access computers or other information is usually a good idea.

Putting together a disability statement prepares you for the moment in the job-hunting process where you'll need to both emphasize your skills and potential and put an employer's concerns to rest. Using plain language to describe your ability to accomplish job duties or meet general goals is important.

You can talk about your disability by explaining how you will perform the job duties in question, or by describing how you have performed similar jobs in the past. It can be helpful to relate possible accommodations to specific job duties and to think about what the employer will want or need to know. Disclosure during the employment process is not an appropriate time to lecture someone about having a
disability. Rather, your disability statement is an opportunity to promote yourself and to help a potential employer recognize how you will be a valuable member on their team.

\textbf{Tips}:

When you describe your disability or impairment, always use positive language, simple terms and phrasing, and include functional implications. Here are some examples:

``I have an eye condition that limits what I can see. It's like looking through a straw. I have to scan or look around more because of this loss of peripheral vision. I can look at you and see your face, but I do not see the rest of you or the surroundings.''

``I use screen reading software called JAWS. It reads information from the screen to me. I use one earphone for listening to my screen reader; I can use the other ear to use the phone or listen to other information.''

``Because of my visual impairment, I am unable to drive, so I use the bus system to get around. I took the bus here today. If the bus is not working, I have other methods of transportation.''

``I use a device that enlarges paper documents to allow me to see them easily. Other documents can be given to me electronically or I can scan them into my computer.''
``As you can see, I have great technology skills and I am very creative and will be able to meet the duties assigned to me.''

\textbf{KEY TAKE-AWAY}:

When you describe your disability or impairment, always use positive language, simple terms and phrasing, and include functional implications.

Example: ``I use screen reading software called JAWS. It reads information from the screen to me. I use one earphone for listening to my screen reader; I can use the other ear to use the phone or listen to other information.''

\pagebreak \subsection*{2-2\quad Assignment: The Disability Statement}
\addcontentsline{toc}{subsection}{2-2\quad Assignment: The Disability Statement}

Write your disability statement. Think about which of these areas you need to address:
\begin{itemize}[leftmargin=*]
\item Travel to and from work (reliability)
\item Completion of routine tasks
\item Technology (can offer possible demonstration or website to view a demonstration)
\item Independence (complete tasks on your own)
\end{itemize}
Use these points to create your disability statement, memorize it, and then practice reciting it to a friend, family member, counselor, or teacher. Get started!
\pagebreak \subsection*{2-3\quad Example Assignment: Joe's Disability Statement}
\addcontentsline{toc}{subsection}{2-3\quad Example Assignment: Joe's Disability Statement}
I am visually impaired, and I have an eye condition called RP or Retinitis Pigmentosa. This eye condition basically has deteriorated my retina from the outside to the inside. I see through a small ``tunnel'' of vision in each eye. The vision I have in the center is good usable vision. I can read with magnification on my computer by utilizing special software. I also use electronic speech through a screen reading feature on that software with one earphone to allow me to use the phone at the same time.

I use a desktop device to magnify print materials or scan them and read them on my computer. I also carry a small handheld magnifier in my pocket, just in case. I have great computer skills that will allow me to be an asset to your business. I can read braille as well, but typically read print. I can demonstrate some of the technology that I use, so that you get a better idea, or I can direct you to a few websites that show videos of it being used.

I know as a human resources representative that I will be organizing a lot of documents. I tend to scan them into my computer and save them in a file system. I always back up my information and feel electronic copies are the easiest. I am able to organize print as well. I label folders with braille labels I will provide. The labels are similar to a piece of tape with some bumps on it. I can show you an example.

Human resources personnel travel once in a while, and I know that will not be difficult for me. I pride myself on being punctual and reliable. I utilize public transportation to travel and have not had any issues getting any around. If a bus is delayed, I will use Uber or Lyft or find another method of transportation. I have built up a great transportation network. My vision is much better during the day than at night or in dark rooms. I use a white cane to help me travel efficiently through environments. I am very confident in my skills and think that I would be a great addition to your business. If you have any questions, please ask and I can answer them for you.
  
\pagebreak \section*{Module 3:	Creating a Personal Data Sheet}\addcontentsline{toc}{section}{Module 3:	Creating a Personal Data Sheet}\extramarks{Module 3}{COURSE 3: Finding Employment}
\noindent\makebox[\textwidth]{\rule{\linewidth}{0.4pt}} \etocsetnexttocdepth{4} \localtableofcontents 
\noindent\makebox[\textwidth]{\rule{\linewidth}{0.4pt}} 


\pagebreak \subsection*{3-1\quad Lesson: Developing a personal data sheet of basic information to support the job application process}
\addcontentsline{toc}{subsection}{3-1\quad Lesson: Developing a personal data sheet of basic information to support the job application process}
In the early stages of the employment process, most employers require the same preliminary information from every applicant. It's important to have this information readily available, accurate, and organized when applying for a job.

A good way to ensure that you're prepared is to develop a Personal Data Sheet that you can refer to whenever you need to fill out an application. Having this information organized in one place will help you complete job applications efficiently and meet important application deadlines. This information will also help you build your resume as you will need to include most of the same data on your resume.

While employers may ask for a different order or use different words to describe this information, they will almost certainly require data from the following seven categories:

\textbf{Personal Information}:

The category includes your full name, current address, past addresses, current phone number(s), email address, date of birth, place of birth, citizenship information, and parents' or guardians' names.

\textbf{Education and Training}: 

List the high school and college you attend or attended with the most recent first. Include the name, address, and phone number of the school and the years of your attendance at each. If you've graduated, include the year of your graduation and specify if you earned a diploma or degree. Include your GPA (unless it’s below average) and any honors you received at each school.

\textbf{Employment History}: 

In this section you should list the locations of previous employers starting with the most recent. For each position include the following: name of the business, address, phone number, your position and duties, your salary at hire (per hour, per month, or per year), salary at the end of employment, the dates of employment, your supervisor's name and phone number, and the reason you left the job.

Keep in mind, employment history can tell an employer a few things about you. If you're a recent graduate, it's understandable you may not have a long work history. If you've been out of school a while and don't have a current work history, this may prompt some questions from a potential employer.

If your work history has many different businesses listed, but you only spent a short period at each, a potential employer may wonder about your loyalty or the quality of your work. Because filling a position and training a new employee requires an investment of an employer's time and effort, employers want to make sure that whomever they hire will stay in the position long enough to see a return on that investment.

In the employment history section of your personal data sheet, on any application, and during any interview, it's important to carefully phrase your reason for leaving each job. You should never lie, but you should think about how you can phrase things so that you are seen in the best light possible.

Never speak negatively about past employers, as the business world can be a small network, and information can be spread through friends and ``friends of friends''. You could easily and quickly damage your reputation in your local community, making it even harder to get hired. As you're developing your personal data sheet, try to think about your work history from the employer's perspective and start to prepare answers for questions that may arise.

\textbf{Community Service/Volunteer Work}: 

Use this category to list any internships, community service, or volunteer work you have done. Include the name of the organization, the address, phone number, contact person (for verification of your hours), the number of hours you donated, the dates you volunteered, and the duties and skills you acquired or learned as a result of your donation of time.

\textbf{Awards}:

Keep a detailed log of the awards you receive while you are in high school and/or college. Awards demonstrate examples of your value to an employer. Do not describe an award using abbreviations or jargon which may be unfamiliar to an employer.

\textbf{Special Skills/Additional Certifications}: 

It is essential for you to list any particular skills that may make you more valuable or qualified. If you were captain of the golf team for two years, you may want to list your leadership skills as an example.

Additional examples of information you may want to include are your knowledge of foreign languages, computer programming languages, special certifications, or additional training. Keep in mind if you state you are proficient in using a software program, you are stating you are an expert and should be prepared to demonstrate your skill level to a potential employer if asked.

\textbf{References}: 

Include the name, job title, address, and preferred contact information of at least three people you've worked with or known well who can vouch for your work ethic and potential to be a valuable employee. Include your relationship to each of your references (supervisor, college professor, etc.) as well as the duration you have known your reference.

Your references should be reputable and be able to express why you would be a good candidate for a job. It's best for your references to be people with whom, or for whom, you have worked, volunteered, or done an internship. If you don't have an employment history yet, teachers or professors you have worked with often can also be good references. Avoid providing family members or friends as references.

Before including someone on this list, ask them if they would be willing to be an employment reference for you. If they are, ask for a written letter of recommendation (some employers will ask for such letters), preferably on business letterhead. Also ask them for their preferred method of contact and make sure that method is what you include on your personal data sheet.

\pagebreak \subsection*{3-2\quad Assignment: Personal Data Sheet}
\addcontentsline{toc}{subsection}{3-2\quad Assignment: Personal Data Sheet}
Open a blank document and create your Personal Data Sheet. If you have already created a Personal Data Sheet, then review and update it using the tips provided to you in this lesson.
 
 
\pagebreak \section*{Module 4:	Building a Resume}\addcontentsline{toc}{section}{Module 4:	Building a Resume}\extramarks{Module 4}{COURSE 3: Finding Employment}
\noindent\makebox[\textwidth]{\rule{\linewidth}{0.4pt}} \etocsetnexttocdepth{4} \localtableofcontents 
\noindent\makebox[\textwidth]{\rule{\linewidth}{0.4pt}} 


\pagebreak \subsection*{4-1\quad Lesson: Developing a crucial job search tool: your resume}
\addcontentsline{toc}{subsection}{4-1\quad Lesson: Developing a crucial job search tool: your resume}
Resumes are necessary in the job and internship seeking process. Regardless, if you are a teenager in high school with no work experience or a recent college graduate with experience holding a few jobs, it's important to always have a high quality and current resume prepared and on hand for whenever a potential employer might ask for one.

Your resume is a tool you can use to showcase your skills, abilities, and accomplishments. It's an advertisement of who you are and should convince an employer you'd be an asset to the place of employment as an employee or intern.

Most employers will formally require submission of a resume at some point in the application process. Even when it's not formally required, most businesses will be pleased to accept a resume accompanied by your job application when you are applying for a job or internship. Your resume may be your best opportunity to sell yourself to an employer before an interview, so it's a good idea to provide one, even when it's not specifically asked for.

It's normal to feel a bit intimidated by the resume development process. If you've been thoughtfully completing the assignments in this course, you've already done much of the legwork required to create a solid resume.

\subsubsection*{What Does a Resume Include?}

Resumes typically have similar categories to those in the Personal Data Sheet such as personal information, educational and work history. Resumes may also include sections for awards, certifications, honors, special skills, and references. Unlike the listing in your Personal Data Sheet, the work history section will typically include an additional description of each of your jobs in order to provide a potential employer with a summary of the kinds of work accomplished. Important note: Never include your social security number on your resume.

\textbf{General Format}

It's important to consider resume formatting. Make sure you use consistent formatting that is easy to read. You want your resume to visually appeal to an employer who might have several resumes to review.

The font, spacing, and overall appearance of your resume are important to consider. Your resume should include the use of typographic emphasis such as bold, underline, and italic so the employer can quickly locate specific information about you. For instance, the headings on your resume should be typed in all caps to delineate new categories of information. In addition, bullets are often used in resumes to list important facts.

The following are key headings and the types of relevant information you need to include on your resume:

\textbf{Personal Information or Resume Heading}

This information is entered at the top of your resume and includes your full name, phone number, address, and email address. Because this identifies yourself to the employer, this information is typically a font size larger than the other text in your resume and is often centered on the page.

The goal is for your name to stand out to the employer. Keep in mind your email address should be formal and standard such as your first initial and last name or your first name and last name. If necessary, create a new email address and use it just for applying for jobs.

It might seem like a small detail, but in a job search it's important to always present yourself in a professional manner. Employers will not hesitate to dismiss a resume with an inappropriate email address.

\textbf{Objective}

The objective is a statement which addresses what you are trying to accomplish or what you hope to get out of the job. Your objective may change depending on the type of job you are applying for so be sure to update this section of your resume when you apply for different jobs. Your objective should be clear and to the point such as ``To gain part-time employment as a stocking clerk''.

\textbf{Summary of Qualifications}

Your qualifications are skills and attributes you possess that can come from your experiences in school, volunteering, extra-curricular activities, training, etc. An example of a qualification you might list is as follows: ``Excellent written and verbal communication skills''. List qualifications which are necessary for the job you are applying for.

\textbf{Education}

List the most recent and subsequent places you have studied at, the dates you attended, and specify what you were awarded such as a diploma, master's degree, etc.

\textbf{Work Experience}

Provide your current or most recent place of employment first, followed by additional places of employment in continued chronological order. Include the dates of your employment as well as key duties you performed.
Always provide the same pieces of information for every job you list.

\textbf{Community Service/Volunteer Work}

Community service and volunteer work are great things to add to your resume especially if you do not have any paid experience to list. Listing volunteer experience on a resume indicates to an employer you take initiative, are self-motivated, willing to try new things, and that you support your community.

\textbf{Honors and Distinctions}

Under this heading accentuate your awards, certificates, and other credentials. Do not use abbreviations or terminology your hiring manager may not understand. This section can help you stand out and give you an edge on others applying for the same job.

\textbf{Activities}

If used right, the information in this section of your resume can help set you apart. Your activities can be professionally relevant as they will demonstrate your potential for leadership and teamwork. Some activities are not appropriate or relevant to list on a resume' such as reading or watching movies.
 
\textbf{Additional Skills}

This is an area to stress skills you have not highlighted on another part of your resume.

\textbf{References}

If your resume has not exceeded the recommended maximum length of two pages, include references on your resume. If you have exceeded the recommended length, notate under the reference heading that ``references are available upon request'' and have a separate reference page prepared to share.

Keep your references as professional as possible. In other words, it would not be appropriate to list your neighbor or your sister. Instead, list past supervisors, teachers, coaches, clients you have babysat for, etc. Overall, you want to list people who can attest to your reliability, your work ethic, and character as a person. Be sure you contact your references for their permission to list them on your resume as well as to find out how they would prefer to be contacted by a potential employer.

\textbf{Length}

For most job seekers in the early stages of their careers, a one-page resume will be the normal size required and preferred by employers. Employers often have dozens or even hundreds of resumes to review for a single job. They may not have the time to read through a long resume to make sure they've caught all of the important points.
One of the most critical parts of resume development is to make sure that you have made it easy for the potential employer to see everything about you very quickly. You might find that your first draft of your resume is longer than a single page. In that case, make sure your writing is as clear and to the point as possible. Then, take a look at your font size, margins, and other formatting options and make adjustments so your full resume fits completely and legibly on one sheet.

As your career progresses and your work history grows, longer resumes will be acceptable and expected.

 \textbf{Accuracy}

Never lie on your resume. Businesses regularly perform fact-checking on applicants before hiring. If it appears you have misrepresented your accomplishments or the facts of your past employment, your application will be dismissed, and your reputation will suffer. If you're hired and it's later discovered that you lied or misrepresented yourself on your resume or application, your employment may be terminated.

\textbf{Key take-away}:
\break \textbf{Always be accurate and customize our resume for each job!}

\textbf{Customization}

Once you've developed a solid master resume, it's a good idea to customize each resume you submit to suit each specific job or employer.

Use keywords found in the job description to highlight in your resume.

Adjust and edit your work history based on the position for which you're applying.

Provide the most detail for the jobs that are most relevant and downplay positions that aren't relevant. If you have gaps in your employment history, make sure you have thought about how to explain them.

Order the headings so you highlight your best qualifications for the position. Let's say for a specific job you have strong educational experience, but not as much relevant work experience. In this scenario, it would be a good idea to put your educational experience above your work experience.

Analyze your resume and decide what category is most applicable to the specific job you're applying for. 

\textbf{Additional Tips on Writing}

Remember to keep writing short and to the point.

For past jobs, use the past tense: ``Trained superiors on how to use the fryer; participated in a training; certified by the United Health Services; trained under an experienced chef; provided customer services; processed user applications''.

For your present job, use the present tense: ``Train coworkers on inventory practices; am responsible for copier maintenance and supply; work with experienced chef; supervise three interns''.

\textbf{Get Feedback from Others}

Print your resume on standard printing paper and ask a person who does not have a visual impairment to review your resume for content and formatting. Editing is important for any person developing a resume, as they are difficult documents to perfect.

\textbf{Keep It Current}

Keep your resume up to date because you never know when a job opportunity will present itself. It's important to make sure your contact information is accurate and appropriate.

\pagebreak \subsection*{4-2\quad Assignment: Building a Resume}
\addcontentsline{toc}{subsection}{4-2\quad Assignment: Building a Resume}
Use all of the information and tips you just learned to create a resume. Be sure to get feedback and edit, then revise your resume for a final, polished product.
 
\pagebreak \subsection*{4-3\quad Example Resume}
\addcontentsline{toc}{subsection}{4-3\quad Example Resume}

Michael Smith
\break 330-993-9877
\break smithm@gmail.com 
\break 446 Winding Way
\break Long Boat Key, FL 33445

\textbf{OBJECTIVE}
\break Reliable and mature high school junior seeking a part-time retail sales position in a company which offers opportunities for employee growth.

\textbf{SUMMARY OF QUALIFICATIONS}
\begin{itemize}
\item Quick and eager learner
\item Ability to work as part of a team or independently
\item Responsible, efficient, and flexible
\item Goal-oriented
\item Polite, respectful, and courteous
\end{itemize}
\textbf{EDUCATION}
\break High School Diploma, George High School, Winterville, FL, Expected June 2018

\textbf{WORK EXPERIENCE}
\break Bagger, Kroger Supermarket, Long Boat Key, FL, Summer 2016
\begin{itemize}
\item Prepared bagged groceries for customers
\item Assisted customers with carrying out bags of groceries
\item Verified prices of items
\item Greeted customers and assisted them with locating items 
\end{itemize}
Movie Theatre Usher, Cinema 12, Long Boat Key, FL, Summer 2015
\begin{itemize}
\item Greeted, directed guests, and collected admission tickets
\item Counted and recorded number of tickets collected
\item Answered questions from guests
\item Paged individuals needed at the box office
\end{itemize}

\textbf{COMMUNITY SERVICE}
\begin{itemize}
\item Contributed 40 volunteer hours to Sal's Thrift Store, Summer 2014
\item Volunteered 50 hours to Suncoast Animal Shelter, Summer 2015
\item Volunteered 35 hours to the Vet's Soup Kitchen, Fall 2016
\end{itemize}

\textbf{HONORS and DISTINCTIONS}
\begin{itemize}
\item National Honor Society Member, 2014- Current
\item Earned George High School Perfect Attendance Award, 2014 and 2015
\item Achieved Honor Roll, George High School, 2014, 2015, and 2016
\item Most Valuable Goal Ball Player, George High School, 2015
\item First Place Science Fair, George High School, 2015
\end{itemize}

\textbf{ACTIVITIES}
\break Drama Club, 2015 and 2016
\break History Book Club, 2014, 2015, and 2016
\break Yearbook Committee, 2014

\textbf{ADDITIONAL SKILLS}
\begin{itemize}
\item Skillful in using Windows and iOS devices
\item Proficient in software platforms including Microsoft Office and Outlook Express
\item 40 words-per-minute typist
\item Fluent in Spanish
\item Certified in First-Aid and CPR
\end{itemize}

\textbf{REFERENCES}
\break Bob Parks
\break Kroger Supermarket 
\break Customer Service Manager 
\break 445 Winding Boulevard
\break Long Boat Key, FL 33445 
\break parksb@gmail.com
\break 330-993-2663

Mary Brown
\break Suncoast Animal Shelter Volunteer Coordinator 
\break 2234 North Shore Drive
\break Long Boat Key, FL 33446 
\break brownm@sasvc.org
\break 330-993-5566

Sheila Conway
\break National Honor Society Coordinator 
\break George High School
\break 446 St. Marks Trail
\break Long Boat Key, FL 33446
\break conways@ghs.org
\break 330-993-8765
 
 
\pagebreak \section*{Module 5:  Finding Job Leads}\addcontentsline{toc}{section}{Module 5:  Finding Job Leads}\extramarks{Module 5}{COURSE 3: Finding Employment}
\noindent\makebox[\textwidth]{\rule{\linewidth}{0.4pt}} \etocsetnexttocdepth{4} \localtableofcontents 
\noindent\makebox[\textwidth]{\rule{\linewidth}{0.4pt}} 


\pagebreak \subsection*{5-1\quad Lesson: Exploring several methods for finding job leads}
\addcontentsline{toc}{subsection}{5-1\quad Lesson: Exploring several methods for finding job leads}
The job search can feel a lot like searching for gold by sifting through wet mud! It can take considerable time and effort, often for seemingly little reward. Along the way, however, you may find rocks that seem worthless but turn out to be very valuable. Job leads can come in many forms, including as information about job openings, either in the form of ads, postings, information from your network, or even rumors or news items. Job leads are a crucial part of the job-seeking process. No job lead is a bad job lead; some just may not pan out for you. It's important to be diligent, patient, and prepared throughout the process.

The methods that most people use to find jobs are typically broken up into three categories:
\begin{enumerate}[leftmargin=*]
\item Networking
\item Cold calls
\item Intermediary
\end{enumerate}
Networking is typically used by persons who have more experience, while cold calling is commonly associated with positions that are lower paying and possibly paid an hourly wage. Intermediary is the method most commonly associated with conducting a job search. Intermediary goes back to searching online employment websites.
Below are several ways to explore or create job leads.

\subsubsection*{Your Personal Network}
The majority of jobs acquired by job seekers are found via contacts and personal networking. Earlier in this course you were asked to think about your personal network and build a pyramid with five levels. We talked about expanding and maintaining your personal network. Now is when that work pays off.

If you have expanded your network and kept your contacts ``fresh'' by staying in touch with the people you've identified as potentially helpful to your job search, it will be much easier to contact them about job leads or possible connections. When searching for job leads, reach out to your network in an organized and appropriate manner.
Be tactful and professional and make contact with a phone call, in person, or via email.
Keep good records of whom you've contacted and the information each contact provided. If you told someone you would follow up with them at a later date, make sure you do so.

\subsubsection*{Social Networking}

A very effective way to network is on social media websites such as LinkedIn, Twitter, or Facebook. Social networking is similar to reaching out to your network by phone, but you have the opportunity to do more to show off your skills and abilities. Learn more about using LinkedIn on the APH CareerConnect and NSITE Connect websites.

\textbf{Always Thank Your Network}

Be appreciative of the people who help you find job leads, whether or not their advice leads to a job. Take the time to thank everyone who helps you by phone, mail, or email. This is important etiquette and will pay off in keeping your network ready and willing to help you whenever they hear of something relevant to your career. If you end up getting a job from one of these leads then you should do something more than a thank you, such as a gift or taking them out to lunch.

Remember: It's important to put in the effort planning your research and developing your job leads, particularly if you're looking for a job in a tight or highly competitive job market.

\subsubsection*{Professional Organizations and Associations}

Many fields have professional organizations you can join in order to access job postings and employment information specific to that field or industry. Companies may pay these professional organizations to post jobs on their websites.

These organizations also may have an email list that is specific to professionals working in the field. Often, organizations and companies will send announcements of job openings to these lists because they know that the recipients are interested in the field.

You may also be able to email the list to see if any of the professionals working in your area have open positions. You can put yourself out there and sell yourself on these lists, but make sure that you know the etiquette for each list before you start posting. Be careful giving out personal information, and always remember to give a method to respond to your inquiry. Professional organizations can have both a national presence and local affiliates or chapters in each state or even certain cities.

\subsubsection*{Conferences, Workshops, and Meetings}

Through your research, you might find announcements for conferences, meetings, workshops, or networking events. These types of gatherings are great places to network, find job leads, and learn more about the current state of the field you're interested in. When attending these events, be prepared, creative, professional, and outgoing. Dress professionally and attend with a game plan to network. Have several copies of your resume, along with a business card or something that you can give to the people you meet so they will remember you.

\subsubsection*{Career Centers and Job Fairs}

We covered career centers and job fairs earlier in this course and they should be considered viable tools for finding job leads. Find out if you have access to career centers and job fairs through local colleges or universities and take advantage of these resources if available.

\subsubsection*{Employment Centers}

Employment Centers are often underutilized resources when it comes to job leads. Employment centers in fact offer a variety of services that can be useful to a job seeker, such as vocational evaluation, skill training, resume review, or possible employment connections.

\subsubsection*{Employment Websites}

There are many employment search websites. Some are more specialized (by industry or experience level, for example) than others. Many businesses post jobs on these sites. You can use these large sites to survey the businesses you are interested in and find out how they publicize their job openings. These sites can also lead you to find other job sites used in your field of interest.

It's a good idea to be skeptical of jobs posted online that sound too good to be true. There are a number of ``work from home'' and other scams commonly found online on all of the major lob listing sites. Another way to find out if something is fraudulent is to do a search for ``scam'' plus some of the information from the listing and see what you can find in the results.
 
\subsubsection*{Accessing Federal Jobs}

Federal agencies have two job application methods available for people with disabilities: competitive and non- competitive placements. Job applicants must meet the specified qualifications and be able to perform the essential job duties with or without reasonable accommodations.

Jobs that are filled competitively are advertised on \href{https://www.usajobs.gov/}{USAJOBS}. USAJOBS is the official job-posting site used by the U.S. federal government. There are approximately 16,000 jobs available on the site each day. Registering on the website allows one to apply to the federal jobs. This takes some time but is worth the effort. The website allows you to select notifications of job advertisements related to key words. USAJOBS is a tremendous resource that all people with disabilities seeking competitive employment should explore. Jobs filled non-competitively are offered to those with disabilities and who have appropriate documentation as specified under the provisions. For more details on these processes, please visit the \href{https://www.opm.gov/policy-data-oversight/disability-employment/faqs/?page=6}{U.S. Office of Personnel Management}.

\href{https://www.dol.gov/agencies/odep}{The U.S. Department of Labor Office of Disability Employment Policy} (ODEP) website offers useful connections to resources for self-employment, youth employment, employer advisement, the latest disability policies, and more. This office advises the U.S. Department of Labor and other government agencies on employment issues regarding people with disabilities.

\subsubsection*{Libraries}

As we've already established in earlier lessons, libraries are extremely good sources for job research, and they're often underutilized and underappreciated sources for job leads. You can visit your local library and find out if they have any resources for an employment search or you can use their computers for Internet access.

It may be helpful to have someone with you to help use printed resources, but otherwise libraries do have staff members who can assist you. If you know that you will need assistance with your search and want to learn about the resources available, it would be a good idea to schedule an appointment with a staff member or librarian. Additionally, many of the search tools that you'll find at the library are available online and can be accessed from home.

\subsubsection*{Create Your Own Leads: Cold Calling}

Creating your own job leads is usually done by ``cold calling.'' Cold calling involves calling organizations where you're interested in working or that offer the type of position you are interested in holding. This method involves calling companies with no introduction or prior connection, and without responding to a specific job listing. Cold calling is probably the toughest method to find a job lead because you must build a relationship quickly in order to convert the call into a meaningful contact.

First, you must identify companies you'd like to cold call. Search online or ask friends and family for ideas. Once you've found a business you're interested in, visit their website and look for a human resources department contact and links such as ``employment opportunities'' or ``job opportunities.'' More and more companies post their job openings online. Maybe there is a position posted that you're interested in, and you can respond to it.

Even if an organization doesn't have jobs posted on their site, it can be worthwhile to cold call them. Often businesses have jobs that are about to become open but are not yet posted. Some businesses may not post jobs or may not be quick about posting their opportunities. Often businesses are willing to hold on to applications or resumes until a relevant job opens at a later time. This is something that you should ask a human resources or business representative when you call.

It's important to note that the employment divisions of companies can go by many names: human resources, personnel department, hiring division, etc. Smaller businesses may have a single person who handles their hiring.

When cold calling businesses, you may find that some organizations only accept applications on certain days and/or at certain times. For instance, restaurants will often reserve a slow day or a low traffic time of day for applicants to come in and meet with staff. Always respect these schedules.

Cold calling is typically more effective for lower wage jobs. This method is not as common for higher wage jobs, but it has happened for persons. It is just another method of job search and more commonly associated with jobs that are paid on hourly rate rather than a salary.

\subsubsection*{Stay Aware When You're Out And About}

Some small businesses hang signs in their windows to advertise they are hiring. If you are interested in working in a small business or store, check the windows by the entry door to see if there is a ``Help Wanted'' or ``Now Hiring'' sign hanging there. If there is, go inside to inquire about the positions they are looking to fill, or call them when you get home to find out more.

\subsubsection*{Bulletin Boards}

Online and physical bulletin boards can be great job lead resources. Physical bulletin boards with job postings are becoming rare, but you can still find them on some school campuses and at local businesses. If you are a college student, most colleges have databases or job posting web pages that list jobs on or off campus.

Message boards or job posting lists on company and organization sites can also be good places to find leads. It's a good idea to be skeptical of jobs posted online that sound too good to be true. As with job listing sites, be aware of possible scams and do your research before pursuing something that sounds too good to be true.

\pagebreak \subsection*{5-2\quad Assignment: Finding Job Leads}
\addcontentsline{toc}{subsection}{5-2\quad Assignment: Finding Job Leads}
An action plan can be a very useful tool when conducting a job search or trying to accomplish any specific goal.

\subsubsection*{Part 1. Develop a Job Lead Action Plan}
A Job Lead Action Plan is a list of the methods you will use and the steps you will take to develop your job leads. The list is a document formatted as an outline. First, list the methods you will use. Then, beneath each listed method, detail the actions that you plan to take. List the people, organizations, and businesses you will contact, and include the method(s) you plan to use to contact each of them. Update your action plan as you receive responses.

\subsubsection*{Part 2. Put Your Job Leads Action Plan to Work}

As you work through your action plan, create a Job Leads List with two categories. The first category will be ``leads for jobs that have openings'' and the second will be ``leads for jobs without openings.'' As you find out about job leads, list each under the appropriate category. Once you've completed your action plan, order these lists based on your interest in each lead.
 
\pagebreak \subsection*{5-3\quad Example Assignment: Josie's Job Leads}
\addcontentsline{toc}{subsection}{5-3\quad Example Assignment: Josie's Job Leads}
\subsubsection*{Part 1. Methods}

\textbf{Personal Network}

I will contact, via email, the following members of my network and ask for any information on jobs in the digital communications field, with an emphasis on website development:
\begin{itemize}
\item John Smith
\item Kerry Hartford
\item Akiko Tsuomo
\item Lynn Pitchkin
\item Aggie
\item Mark
\item Professor Danforth
\item Lynn Paltrow 
\end{itemize}
\textbf{Library}

I will make an appointment with a librarian at the University of Kentucky to see what resources they have available for finding job leads.

\textbf{Professional Organizations and Associations}

I found three organizations that I'd like to look into:
\begin{itemize}
\item Professional Association for Web Developers
\item University and College Web Professionals
\item Freelance Web Designer Network
\end{itemize}
I'm going to see if any of these associations have email lists and bulletin boards and f ind out what it takes to join and/or gather the information.

\textbf{Conferences}

The University and College Web Professionals association is having a conference in a month, and I am going to look into how much it costs to attend. If I can afford it, I am going to sign up and begin preparation for networking during the conference.

\textbf{Cold Calling}

Right now, there are three local businesses that I know I would love to work for:
\begin{itemize}
\item DBG design
\item University of Kentucky, Louisville, Communications and Student Recruitment
\item Fork and Spoon Web Services
\end{itemize}
I am going to contact each and talk to them about their hiring processes. I'm going to see if I can send them a resume now, even if they don't have any positions open. I'll ask them how long they hold resumes, and also ask them if I might come in to do an informational interview with someone.
 
\subsubsection*{Part 2: Job Leads}
 \textbf{List Leads for Open Jobs}
 \begin{itemize}
\item Site developer II, Fork and Spoon Web Services
\item Assistant web developer, Chrion Associates via Akiko Tsuomo
\item Communications manager, XO Communications
\item Digital asset manager, Getty Images-Lexington via Professor Danforth
\item Digital asset assistant, Getty Images-Lexington via Professor Danforth
\item Web intern, Lexington Post 
\end{itemize}

\textbf{Leads for Jobs Without Openings}

\textbf{DBG Design has the following positions in its employment structure}: 
\break Web Developer
\break Assistant Web Developer
\break Web Manager
\break Digital Communications Assistant
\break Assistant Manager
\break Digital Asset Management
\break Web Intern (college credit only).

\textbf{Univ. of Kentucky Louisville} has a huge quantity of communications positions, and their human resources department says that the best way to learn about open positions is via their HR Website. Right now, there aren't any open positions, but they told me I should look every week.

\textbf{I've signed up for the listservs and have reviewed the bulletin boards for Professional Association for Web Developers}, University and College Web Professionals, Freelance Web Designer Network. Right now, nothing is available in my area, but it seems like the boards are active and the job postings are updated regularly.
  
\pagebreak \section*{Module 6:	Utilizing Job Leads}\addcontentsline{toc}{section}{Module 6:	Utilizing Job Leads}\extramarks{Module 6}{COURSE 3: Finding Employment}
\noindent\makebox[\textwidth]{\rule{\linewidth}{0.4pt}} \etocsetnexttocdepth{4} \localtableofcontents 
\noindent\makebox[\textwidth]{\rule{\linewidth}{0.4pt}} 


\pagebreak \subsection*{6-1\quad Lesson: Following up on job leads}
\addcontentsline{toc}{subsection}{6-1\quad Lesson: Following up on job leads}
Now that you have your list of job leads ordered based on your interest, it's time to follow up on them. Following job leads takes tact, poise, patience, and persistence. During this process you may feel like you're traveling through a maze where you have to explore many paths until you find the one that leads you to the other side. It might take some time but remember: the right path could be the start of a career!

\subsubsection*{Preparation}

Preparation is the key to success. It's time to do a little research about each job lead you have. The idea is to have as much information as possible before contacting the employer so that you can make a good first impression. Take good notes and keep these notes organized. Remember to keep updating your Resource Log. All of this information can be useful if you get an interview.

If you found the lead through a contact, gather as much information from the contact as possible, while still being polite. Remember to thank your contacts!
Research the business for which you have a lead.

\subsubsection*{Explore the business’ website.}
Find out about the products or services they offer.

Do a general search for the business on the Web. You will likely find a variety of information available on the Internet.

Find any current news articles that mention the business and read them. It's important to be up-to-date with the business' current issues.

Find out who the business’ major competitors are and a few basics about them.
Search the Internet to see if there are any current trends or major happenings in the field or market in which the business operates.

Practice what you will say by role playing with a friend, sibling, or parent.

\subsubsection*{Contact}

Once you've done your research, you are to contact the business and follow up on the job lead. Always note the people who you meet at an organization whether via phone, email, or in person (this is a skill that can translate to many parts of your life.) Write thank you notes to anyone who helps you with this process.

Be polite when speaking to anyone. Proper manners go a long way when trying to create new opportunities.

Be persistent when pursuing a job lead, but not overly aggressive. You want them to know that you are eager to work for them, but you don't want them to feel harassed.

\subsubsection*{Questions you can ask:}
\begin{itemize}[leftmargin=*]
\item Who is their human resources/personnel contact?
\item How or where do they post their job openings?
\item How often do they post their job openings?
\item What is their application process?
\item Do they keep applications on file for the future?
\item How long they keep applications on file before tossing them?
\end{itemize}
Most people, no matter their level of experience, are a little nervous when contacting potential employers. Following up on job leads can be an intimidating process, but you will become more confident with it the more you practice.

\pagebreak \subsection*{6-2\quad Assignment: Utilizing Job Leads}
\addcontentsline{toc}{subsection}{6-2\quad Assignment: Utilizing Job Leads}
You now have a list of job leads and several tips on how best to follow up on them.
Choose a lead to investigate and contact. It may be best to start with a lead that you feel is either a weak lead or one that is less desirable to you. This will allow you to practice before you move on to the more desirable and reliable leads.

Once you have done your background research and preparation, role play how you will contact a job lead. Find a friend or relative who is willing to work with you on several scenarios and practice what you might say and how you might say it. Practice several times over several days if you can.

Once you feel that you've practiced enough, contact your lead and have a conversation with them. Submit the following to your portfolio:
\begin{enumerate}[leftmargin=*]
\item The job lead you are following up on for the assignment
\item A description of how you will contact that job lead
\item A script of what you will say when contacting this job lead
\item After contacting the job lead and initiating a conversation, create a description of your experience
\item Once you complete the contact, think about the areas you'd like to improve. What went well? What didn't?
\end{enumerate}

\pagebreak \subsection*{6-3\quad Example Assignment: Josie's Job Lead Follow-Through}
\addcontentsline{toc}{subsection}{6-3\quad Example Assignment: Josie's Job Lead Follow-Through}

The lead I am following is a position I heard about from Professor Danforth: Digital asset assistant, Getty Images-Lexington.

I will be calling Marion Fishman, a colleague of Professor Danforth's by phone and say the following:

Hello, Ms. Fishman, my name is Josie Armentrout. Professor Cosmo Danforth gave me your contact information because he thought I would be a good fit for the Digital asset assistant position. I was hoping we might talk for a moment about the position. Is now a good time? (If not, ask if there is a better time and schedule a call.)

Thanks very much for your time. I really appreciate it. I'm very excited to apply for the position. I have a resume right here. Can you tell me the best way to go about getting it to you?

Do you have the name of someone I can speak to directly at Human Resources?

Can you tell me a little bit about the type of person you think is best suited to this position?

Is there any other advice or guidance you would give me that would increase my chances for working at Getty Images?

Thanks so much for your time. This conversation has been very helpful, and I appreciate it.

In general, my contact with Ms. Fishman went well. I was nervous and sounded a little funny on the phone, but I managed to ask the questions I needed to. Thanks to my script and all the practicing I did I only rambled a little bit. She was nice and gave me a lot of good information. I'd like to be more poised and less nervous the next time I call a contact.
 
 
\pagebreak \section*{Module 7:	The Cover Letter}\addcontentsline{toc}{section}{Module 7:	The Cover Letter}\extramarks{Module 7}{COURSE 3: Finding Employment}
\noindent\makebox[\textwidth]{\rule{\linewidth}{0.4pt}} \etocsetnexttocdepth{4} \localtableofcontents 
\noindent\makebox[\textwidth]{\rule{\linewidth}{0.4pt}} 


\pagebreak \subsection*{7-1\quad Lesson: Writing an effective cover letter}
\addcontentsline{toc}{subsection}{7-1\quad Lesson: Writing an effective cover letter}
Now that you have identified several jobs that you would like to apply for, it's time to write a cover letter. The goal of a cover letter is to persuade an employer to review your resume. A resume can make a good case for your relevant preparation for a job, but it will not tell the employer why you would be a great candidate for a job. It also will not convey much of your attitude and personality.

A cover letter gives you the opportunity to point out specific experiences not fully covered in your resume that might make you a valuable or exceptional candidate or employee.

A cover letter is a formal letter submitted as an accompaniment to a resume. A standard cover letter should run no longer than three paragraphs, and be simple to read, clear, concise, typed without errors, and formatted properly.

\subsubsection*{Cover Letter Tips}
\textbf{Style tips and other basics:}
\begin{itemize}[leftmargin=*]
\item Use formal, professional language-don't use slang or casual phrasing or vocabulary.
\item Be polite.
\item Sell yourself: highlight your strengths and be positive.
\item Use a standard and legible font such as Times New Roman or Arial.
\item Use a word processing system with spelling and grammar checks.
\end{itemize}
Have someone else carefully review your cover letter for mistakes and phrasing and formatting issues before you send it. It's a good idea to get the assistance of a person without a visual impairment to review your cover letter and resume.

If you are emailing your letter and resume to the company, you should both copy and paste your letter into the email and attach it as a document along with the resume. Unless the employer requests otherwise, the cover letter and resume should be separate documents.

The cover letter should be no longer than three paragraphs and it should fit on one page.

Make sure the email address that you use in all correspondence, and the email address that appears in all of your documents is something formal like your first initial and last name.

\textbf{Use formal letter format}:

Single spaced, 10-point font, left justified, one-inch margins. 

Return address on top right of the page.

Address of business or employer below the return address, but on the left side of the page. Leave a line empty.

\textbf{Greeting}: ``Dear Mr. Vicious:'' or ``Dear Ms. Torra:''

It's always better to address your cover letter to someone specific. If you have not spoken with an individual at the company you are applying to, call the office and ask for the name of the human resources director or the name of the person who reviews applications. If for some reason you are unable to get the full name of a contact, you may as a last resort address your letter ``Dear Human Resources Representative:'' or ``To whom it may concern,''

Leave a line empty.

\textbf{First paragraph}: Express your interest in the specific position at their company. Make sure to include the title of the position as it appears on the job posting and, if any reference numbers appeared on the posting, include those as well. If you were referred by someone in your network or at the company, mention them by name. Mention specifics about the job or company that interest you; tie in personal experience or something that shows the extent of your research into the company or the job.

\textbf{Second paragraph}: Describe how you are a good candidate. Be specific and highlight the most important parts of your resume, or something your resume can't cover. This is the time to describe why you are a great fit for the job. What can you bring to the position? What can you bring to the company? Think of skills, personality traits, knowledge, training, experience, enthusiasm, passion, strong work ethic, etc.

\textbf{Final paragraph}: Thank them for taking the time to consider your application and state that you hope to hear from them soon.

Leave a line empty 

Close with ``Sincerely,''

Leave two or three lines of space for your signature

\textbf{Your full name}: ``Daniel Stevens'' or ``Mr. Daniel Stevens''

\pagebreak \subsection*{7-2\quad Assignment: The Cover Letter}
\addcontentsline{toc}{subsection}{7-2\quad Assignment: The Cover Letter}
Use the tips that were provided in the lesson to write a cover letter that sells your application to an employer for a specific job you found through your prior work. Use a word processor with spell check if at all possible.
 
\pagebreak \subsection*{7-3\quad Sample Cover Letters}
\addcontentsline{toc}{subsection}{7-3\quad Sample Cover Letters}
\subsubsection*{Sample Cover Letter 1}

To Nathaniel Strechay

Human Resources

I am applying for the Assistant Parts Inventory Manager position advertised on the City Times website. I have always had a passion for automobiles. I am extremely organized and have had prior positions that demonstrate my skills in inventory. As you'll see from my included resume, I have two years of retail inventory and tracking experience for a large retailer. I have a good knowledge of automobile parts and automobiles in general.

I feel that I would be a great fit for this position and would bring enthusiasm and a strong work ethic. I am experienced with computers, databases, and a variety of software. I feel my attributes would add to your already well-established business. Your reputation for great service precedes you, and I would be pleased to contribute to the continued success of your company.

I look forward to hearing from you in the near future. You can reach me at sreagan@genericemailaddress.com or at 201-555-5555.


\pagebreak\subsubsection*{Sample Cover Letter 2}
To Stan Vanderslooth / Huffman \& Huffman

I am writing to express my interest in the Administrative Assistant position posted on monster.com.

This position is a perfect match for my background and skills. I am a graduate of Huntington High School where I took vocational training in executive assistant and secretarial services. I completed a work experience internship at Moises Automotive where I served as an assistant to the executive assistant for three months. I received great on-the-job training there, and I am confident that with this experience I would be an asset to your organization.

I am skilled at using Microsoft Office and a variety of other computer programs that would be necessary for fulfilling the duties outlined in the job posting. I've handled the phones at two other organizations to get experience with different phone systems. I am extremely organized and meticulous about my work duties.

I would love the opportunity to meet with you or other members of your organization to discuss this job in more detail. You may contact me at j.taylormasterson@genericemail.com or contact me at 304-555-5555.

Sincerely

Jason T. Masterson
  
\pagebreak \section*{Module 8: The Application}\addcontentsline{toc}{section}{Module 8: The Application}\extramarks{Module 8}{COURSE 3: Finding Employment}
\noindent\makebox[\textwidth]{\rule{\linewidth}{0.4pt}} \etocsetnexttocdepth{4} \localtableofcontents 
\noindent\makebox[\textwidth]{\rule{\linewidth}{0.4pt}} 


\pagebreak \subsection*{8-1\quad Lesson: Filling out a job application form}
\addcontentsline{toc}{subsection}{8-1\quad Lesson: Filling out a job application form}
When filling a position, employers use standard application forms to gather basic information about the candidates for every job. Applications may ask for much of the information you've included on your resume, but both are usually required in the hiring process. Most of the information you will need to fill out on an application can be found on your Personal Data Sheet and in your resume.

\textbf{Seek Information}

You will first want to find out how the employer usually handles-or prefers to handle-applications. Are the applications available online, by mail, or will you need to go into the office to pick one up? Do they accept applications only on certain days of the week or month or during certain hours? Do they prefer that candidates fill out the application at the office? Is it possible (or required) to complete and submit the application online?

\textbf{When/If You Visit}

When going to a potential employer's office to pick up or drop off an application, have a plan to deal with situations such as
\begin{itemize}[leftmargin=*]
\item assistance traveling to, and orienting within, the location
\item assistance filling out a printed application
\item reading any documents that you may be required to review or sign
\end{itemize}

When you visit, your goal is to be as independent as possible and to make it evident to the business or employer that you are capable of handling any tasks while you are there. It can be tough to negotiate new locations with confidence; it may take some practice. If you need assistance, advise your helper to allow you to address any issues or questions that arise.

Be prepared to explain why you are using assistance. It's important to be direct, yet polite. If using portable assistive technology will enable you to accomplish tasks independently, use it. Technology can be an interest grabber and create an opportunity to educate. People are often curious about and impressed by the effective use of technology, and some may feel that your use of technology demonstrates you are competent and will be able to accomplish job duties.

\textbf{Practice First}

Ideally you will be able to obtain a copy of the application form before you need to submit it so you can practice filling it out. Everyone makes mistakes when filling out applications due to small print, strange formats, and unfamiliar phrasing used by each business. Make a copy of the application and practice filling it out prior to filling out the version you intend to turn in. When you are filling in your final application, be calm, take your time, and be prepared. If you are filling out a paper application, you will need to print legibly or have someone print legibly for you.
 
\textbf{Keep a Record}

Create a record system or journal to list your application submissions. Keep track of whom you contact on what date; when you picked up applications; when you turned in applications; when you followed up with employers; initial contact with employer; method of contact; contact information; whom you interacted with at the employer; and any other information you think is important.

\textbf{Review Samples}

Review the sample application included in this section and familiarize yourself with the kinds of information you'll need to provide.
\pagebreak \subsection*{8-2\quad Assignment: The Application}
\addcontentsline{toc}{subsection}{8-2\quad Assignment: The Application}
\subsubsection*{Part 1: Create an Application Log}
Create a log to keep track of:
\begin{itemize}
\item date of initial contact with employer
\item method of contact
\item contact information
\item whom you interacted with at the place of employment
\item when you picked up an application
\item when you turned an application
\item when you followed up on an application
\item information on all contacts following
\item any other important information you'd like to keep track of
\end{itemize}
Create a sample record and list the important events that would be a part of your application process. 

\subsubsection*{Part 2: Application Practice}
Fill out the sample application

\pagebreak \subsection*{8-3\quad Example Assignment: Josie's Application Log, sample entry}
\addcontentsline{toc}{subsection}{8-3\quad Example Assignment: Josie's Application Log, sample entry}
\textbf{Employer}: Getty Images
\break \textbf{Date of Initial Contac}t: April 22, 2020 
\break \textbf{Method of Contact}: telephone
\break \textbf{Contact Information}: Marion Fishman, 999-444-1112
\break \textbf{Picked up application}: April 24, 2020 (online) 
\break \textbf{Turned in application}: April 30, 2020 (online)
\break \textbf{Follow up}: May 3, 2020, Getty HR office, spoke to Eddie Harcourt on the telephone who said they had received the application, the cover letter, and the resume, and they will be in touch if they want to interview me.
 
\subsubsection*{Employment Application}

\textbf{NAME}
\break First Name:
\break Middle Initial:
\break Last Name:

\textbf{ADDRESS}
\break Address 1 (number \& street):
\break Address 2 (apt., suite, etc...):
\break City:
\break State:
\break Zip Code:

\textbf{CONTACT INFORMATION}
\break Phone Number (Home):
\break Phone Number (Business):
\break Phone Number (Mobile):
\break Email Address:
\break Position of Interest (Title):

\textbf{Hours: (Select one)} 
\break Full-time
\break Part-time
\break Other (Explain below)

\textbf{Comment:}

\textbf{How did you hear about this position?}

\textbf{PAST EMPLOYMENT}

\textbf{(Most Recent)}
\break Position Title: 
\break Business Name: 
\break Business Address:
\break Business Phone Number:
\break Website:
\break Supervisor's Name:
\break Supervisor's Phone Number:
\break May we contact this person? Yes/No 
\break Hours per Week:
\break Start Date:
\break End Date:
\break Reason for Leaving:
\break Your Job Duties:

\textbf{(Past Employment 2)} 
\break Position Title: 
\break Business Name: 
\break Business Address:
\break Business Phone Number:
\break Website:
\break Supervisor's Name:
\break Supervisor's Phone Number:
\break May we contact this person? Yes/No 
\break Hours per Week:
\break Start Date:
\break End Date:
\break Reason for Leaving:
\break Your Job Duties:

\textbf{EDUCATION/ TRAINING/ CERTIFICATION}

\textbf{(Please fill out applicable information) }
\break High School:
\break Year of Graduation:
\break Diploma: (Yes/No)
\break College/Post-Secondary/Vocational
\break Name of Institution:
\break Years attended:
\break Degree or Certification:

\textbf{Applicable Training:}



\textbf{REFERENCES}
(List 2 References)
(Reference 1)
\break Name:
\break City/State:
\break Phone:
\break Relationship to Applicant:

(Reference 2) 
\break Name:
\break City/State:
\break Phone:
\break Relationship to Applicant:

\break I, \rule{2cm}{1pt} certify that the information that I have listed is accurate. I acknowledge that any misinformation provided will eliminate me from employment consideration.

SIGNATURE: 	 

DATE: 	

PRINT NAME: 	

