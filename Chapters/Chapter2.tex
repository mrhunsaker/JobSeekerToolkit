\hypertarget{course2}{}\chapter*{COURSE 2: Career Exploration - Resources and Methods}\label{course2}\addcontentsline{toc}{chapter}{COURSE 2: Career Exploration - Resources and Methods}
\extramarks{ }{COURSE 2: Career Exploration - Resources and Methods}
\noindent\makebox[\linewidth]{\rule{\linewidth}{0.4pt}}
\localtableofcontents 
\noindent\makebox[\textwidth]{\rule{\linewidth}{0.4pt}} 
\newpage
\pagebreak \section*{Module 1: Career Exploration Resources}\addcontentsline{toc}{section}{Module 1: Career Exploration Resources}\extramarks{Module 1}{COURSE 2: Career Exploration - Resources and Methods}
\noindent\makebox[\textwidth]{\rule{\linewidth}{0.4pt}} \etocsetnexttocdepth{4}
\localtableofcontents 
\noindent\makebox[\textwidth]{\rule{\linewidth}{0.4pt}} 


\pagebreak \subsection*{1-1\quad Lesson: Identifying the best research resources to support your job search}
\addcontentsline{toc}{subsection}{1-1\quad Lesson: Identifying the best research resources to support your job search}
Effective research is more about quality than quantity. If you gather a hundred resources, how likely is it that you will have the time to learn about each of them in order to fully use them for your purposes? A better way to frame your efforts is to ask the following question. What tools, media, or resources can most effectively be used to support your job research? Below you'll find a selection of commonly available resources, along with a discussion of what each has to offer. These resources are good starting points for the first phase of your research.
\subsubsection*{APH CareerConnect®}
In the Introduction, you learned about the APH ConnectCenter website, APH CareerConnect, a free online resource for people who want to learn about the range and diversity of jobs performed by adults who are blind or visually impaired throughout the United States and Canada. APH CareerConnect is a great resource for job search information and tips.
\subsubsection*{NSITE Connect}
The right job is out there, regardless of your field or skills. New opportunities for people who are blind, visually impaired, and/or veterans are posted all the time, each with the workplace environment, leadership, and accommodations to help you succeed. Use NSITE Connect to explore career opportunities and check out open positions nationwide shared by more than 40 employers. 

\href{https://nsiteconnect.jobboard.io/}{Access the NSITE Connect Job Board}
\subsubsection*{Libraries}

Libraries are an important resource for any job seeker. At your local library you can find books in audio or other formats, access major online research databases, and find additional information and guidance. Most of the resources at the library are available online.

If you are a student at a high school, vocational school, community college, college, or university, your institution's library may provide access to even more online resources.

Library staff are trained to help you find the information and resources that will support your research. Some universities and public libraries have staff trained to work specifically with persons with disabilities. Find out what is available at your local library and take advantage of whatever resources you locate.

\subsubsection*{Career Centers}
Career Centers help people perform research to support professional goals. Colleges, universities, and vocational schools often have career centers, and many are available to the public. You may have to visit, call, or do some online research to find out what is available to you locally. Keep in mind that many of the career centers around the country maintain robust websites accessible to anyone with an Internet connection.
Career Centers are often underutilized, and most are eager to have visitors.

\subsubsection*{Vocational Rehabilitation Agencies}
Vocational rehabilitation agencies help people with disabilities prepare for entry or re-entry into the workforce. Your local vocational rehabilitation agency will offer a range of programs, resources, and services to help you get to work. The range of programs offered by these agencies varies from state to state, so research your local vocational rehabilitation agency, determine what programs and services you are eligible for, and get registered.

In most cases, these organizations exist to help you become job-ready and find employment. Some may also train you in independent daily living, orientation and mobility, and access technology. These organizations will also know about other available resources in your community and state. To find a local or state agency near you, use the \href{https://aphcareerconnect.org/directory/about-the-aph-directory-of-services/}{APH Directory of Services}.
\subsubsection*{O*-Net Online}
The \href{https://www.onetonline.org/}{O-Net Online} website provides the latest statistics about a wide variety of occupational fields. The site is a part of the U.S. Department of Labor, Employment and Training Administration.

\pagebreak \subsection*{1-2\quad Assignment: Career Exploration Resources}
\addcontentsline{toc}{subsection}{1-2\quad Assignment: Career Exploration Resources}
\textbf{Part 1}: Setting up a Resource Log Before you start investigating research resources in earnest, you'll want to establish a way of organizing the agencies and information you find by putting together a resource log.

It's likely that you've had to document sources when writing papers in school. When it comes to the job search, creating a resource log allows you to quickly find important information, sources, and contacts who provided information that helped you along the way.

Throughout your job search, there will be occasions when you will want to access or remember information that you've already collected. It's important to set up a resource log early so that you can record the information you find as you do your research.

First, create a folder on your computer called 
\begin{itemize}[leftmargin=*]
\item ``Resource Log.'' 
\end{itemize}

Within that folder, create separate documents for each type of information you will be recording. 

\textbf{Examples might be: }
\begin{itemize}[leftmargin=*]
\item ``Contacts,'' 
\item ``Organizations,'' 
\item ``Books,'' 
\item ``Articles,'' 
\item ``Emails,'' 
\item ``Websites,'' 
\item ``Miscellaneous.'' 
\end{itemize}
Make documents for each of the categories that you think you'll need. The key is that the system has to work for you, be easy to refer to and update, and simple to expand.

For each resource, you'll want to capture the type of information most useful for you. You want to make sure you know what the important information is associated with the resource, and also how to locate the resource again. The lists below will get you started, but don't hesitate to add or customize the information you record, based on your own experience.

\pagebreak \textbf{Organizations}
\begin{itemize}[leftmargin=*]
\item Name of organization or agency
\item Address
\item Website
\item Description
\item Resources available 
\end{itemize}
\textbf{Books}
\begin{itemize}[leftmargin=*]
\item Title
\item Author
\item Page or Chapter
\item Notes
\end{itemize}
\textbf{Websites}
\begin{itemize}[leftmargin=*]
\item Web address
\item Company/Organization
\item Section of the site
\item Title of the article or page
\item Author (if attributed)
\item Notes 
\end{itemize}
\textbf{Contacts}
\begin{itemize}[leftmargin=*]
\item First and last Name
\item Organization
\item Job Title
\item Phone Number
\item Email Address
\item Notes
\end{itemize}
\pagebreak \textbf{Part 2}: Now that you have a system in place for keeping track of the information you're going to find in your research, familiarize yourself with the resources in your community and beyond. Where are your libraries, career centers, and vocational rehabilitation agencies? What services do these organizations offer? Fill out a resource log form for each of the resources you look into.

\pagebreak \subsection*{1-3\quad Example Assignment: Thomas's Resource Log:Organizations}
\addcontentsline{toc}{subsection}{1-3\quad Example Assignment: Thomas's Resource Log: Organizations}
\textbf{Name of organization or agency}: Florida State University Library 
\break \textbf{Address}: 123 Tallahassee Ave
\break \textbf{Website}: \href{http://www.lib.fsu.edu/}{http://www.lib.fsu.edu/}
\break \textbf{Notes}: University Library system; access to almost all services is available to community members; limited borrowing privileges
\break \textbf{Resources available}: Librarians, online databases and periodicals, job boards, research assistance, digital references, and job postings

\textbf{Name of organization or agency}: Tallahassee Community Library 
\break \textbf{Address}: 456 Main Street
\break \textbf{Website}: \href{http://www.leoncountyfl.gov/library/}{http://www.leoncountyfl.gov/library/}
\break \textbf{Notes}: Local branch library and online resources through the Leon County Public Library System. 
\break \textbf{Resources available}: DVDs, CDs, digital books, databases, librarians and workshops

\textbf{Name of organization or agency}: FSU Career Center 
\break \textbf{Address}: 987 Franklin Ct.
\break \textbf{Website}: \href{http://www.career.fsu.edu/}{http://www.career.fsu.edu/}
\break \textbf{Notes}: Online and in-person career assistance. Most of their resources are available to me. I have an appointment for an orientation.
\break \textbf{Resources available}: Everything except on-campus and on-line events organized for current students. Job boards, interview preparation, resume review, career guidance, online FAQs and advice.

\textbf{Name of organization or agency}: Florida Division of Blind Services
\break \textbf{Address}: 6578 Belloc Road
\break \textbf{Website}: \href{http://www.myflorida.com/dbs/}{http://www.myflorida.com/dbs/}
\break \textbf{Notes}: I am a client of the Florida Division of Blind Services, which is my state's vocational rehabilitation agency. I am in the transition program and participate in the Lighthouse of the Big Bend's summer and year-round transition programs. I work with my transition counselor at the Florida Division of Blind Services, who helps me make sure that I am getting the services needed, including work experience, and who ensures I know what other opportunities are available. My counselor actually introduced me to this online course.

\textbf{Thomas Malcum's Resource Log}: Websites
\break \textbf{Web Address}: \href{http://www.aphcareerconnect.org}{http://www.aphcareerconnect.org} 
\break \textbf{Company/Organization}: APH ConnectCenter Section of the Site: APH CareerConnect
\break \textbf{Title of Article or Page}: For Job Seekers
\break \textbf{Notes}: Lots of advice, information, and guidance on job seeking and the status of various industries in the country. You can learn about successful professionals and learn how to be organized and in your job search.
 
 
\pagebreak \section*{Module 2:	Starting a Job Analysis}\addcontentsline{toc}{section}{Module 2:	Starting a Job Analysis}\extramarks{Module 2}{COURSE 2: Career Exploration - Resources and Methods}
\noindent\makebox[\textwidth]{\rule{\linewidth}{0.4pt}} 
\etocsetnexttocdepth{4} 
\localtableofcontents 
\noindent\makebox[\textwidth]{\rule{\linewidth}{0.4pt}} 


\pagebreak \subsection*{2-1\quad Lesson: Laying the groundwork for a detailed job analysis}
\addcontentsline{toc}{subsection}{2-1\quad Lesson: Laying the groundwork for a detailed job analysis}
A job analysis is the process through which a job seeker collects information on the duties, responsibilities, necessary skills, growth opportunities, knowledge, and environment and atmosphere relating to a specific job.

A job analysis collects information from a variety of sources and creates a picture of a position that you can use to determine if a job is truly a good fit for you. In order to perform a thorough job analysis, you will need to first identify the job you'd like to learn more about, and then seek out information from online sources, people, and organizations.
At the beginning of your job analysis, you will look for basic information such as: job title, duties, field/profession, qualifications (training, certifications, and required professional experience), salary, and geographic locations (where these jobs are typically available).
\subsubsection*{Job Descriptions}

Job descriptions, usually found in job postings, classified ads, and job boards, are a good entry point to learning about a specific position. A job description is the summary of an organization's expectation for what the job entails; the major duties involved; the types of skills, special training, certification, or degrees preferred or required to perform the job; the reporting structure; wage information; status (full-time/part- time; temporary/permanent); hours; location; and other important information. Understanding the information contained in job descriptions is a central aspect of job analysis.

The types of tasks, duties, and requirements detailed in a job description can tell you a lot. Jobs that may be considered higher level skilled positions are usually associated with higher wages and more extensive experience, training, and/or education.
\subsubsection*{Other Sources of Basic Information}
In addition to posted or published job descriptions, there are other resources that provide both basic and detailed information.
\subsubsection*{Your Resource Log}
Earlier you identified several research sources and set up records for the types of information located at or in each. Review your Resource Log. Where do you think you can find information on specific positions?
\subsubsection*{APH CareerConnect}
APH CareerConnect can be a great resource for job analysis. You can read the article, How to Find the Fastest Growing Industries in America. Often the results of your search will provide detailed career information similar to what you would find in a job description.
\subsubsection*{NSITE Connect}
You can use NSITE Connect to explore information about specific employers, peruse job descriptions, or access additional resources as well as testimonials shared by other jobseekers. 

\href{https://nsite.org/job-board/}{NSITE Connect}

\pagebreak \subsection*{2-2\quad Assignment: Job Analysis}
\addcontentsline{toc}{subsection}{2-2\quad Assignment: Job Analysis}
The goal of this assignment is to begin to methodically collect information about the job that you are most interested in exploring.

\subsubsection*{Choose The Job You'd Like to Analyze}
In Course 1, you did a lot of work to identify your skills, interests, personality, and values. In an earlier assignment you listed five jobs or professions that you thought might match with your work personality.
Revisit that assignment and think about which of those positions seem like the best fit for what you learned in your self-assessment. Choose the one you are most interested in.
\subsubsection*{The Job Information Form}
To help you consider what information is critical to capture at the start of your analysis, a Job Information Form is reproduced below. The form is meant to be used as a guide. Customize the form so that it reflects any job- or field-specific information, or anything else you would like to know.

Not only will filling out a Job Information Form help you get a better sense of each position you're interested in, it will also help you prepare for much of the work you will be doing later in this module. In the next sections, you will be establishing a relationship with a mentor, setting up an informational interview, and finding an organization that will allow you to do an occupational observation. As you learn more about the position, you will want to keep a record of the questions you want to ask the people involved in these next steps.

Begin the assignment by finding a job description or descriptions and flesh out the information with APH CareerConnect, NSITE Connect, and O-Net information-or information you find through library, career center, or vocational rehabilitation agency research. Make sure you take accurate notes on the research you conduct, and don't forget to continue to add to your resource log as you identify sources of information.

\pagebreak \textbf{JOB INFORMATION FORM}:

Common Job Title (s): \hrulefill

Major Job Duties: \hrulefill

Required Education and/or Certification(s): \hrulefill

Required Professional Experience: \hrulefill

 Other Qualifications (special computer skills, languages, etc.):  \hrulefill

 Position this Job Reports to: \hrulefill

 Hours: \hrulefill

 Location: \hrulefill

 Salary: \hrulefill

 Job and/or Industry Outlook: Other Information: \hrulefill

 Questions to ask a worker in this field: \hrulefill
 
\pagebreak \subsection*{2-3\quad Example Assignment: Laura Smith’s Job Information Form}
\addcontentsline{toc}{subsection}{2-3\quad Example Assignment: Laura Smith’s Job Information Form}
\textbf{Common Job Title(s)}: Paralegal, Legal Assistant, Legal Investigator, Patent Agent

\textbf{Major Job Duties}: Assist lawyers via research, document and case preparation. Conduct research to support a legal proceeding. Handle legal correspondence, maintain organization of documents in an established paper or electronic system. Prepare briefs, pleadings, appeals, wills, contracts, etc. Contact witnesses, meet with clients. Investigate facts and law of cases and conduct background research. Direct and coordinate law office activity. File pleadings. Gather and analyze research data, such as statutes, decisions, and legal articles. Ensure law library is up-to-date.

\textbf{Required Education and/or Certification(s)}: Training in vocational school, related on the job experience, or an associate's degree. Usually need one or two years of training.

\textbf{Required Professional Experience}: One or two years of experience; might have a formal apprenticeship structure

\textbf{Other Qualifications (special computer skills, languages, etc.)}: Fluency with computers, hardware and software.

\textbf{Position this Job Reports to}: Lawyer, Partner in a Law Firm, Other Manager Hours: Full-time

\textbf{Location}: Everywhere 

\textbf{Salary}: \$46,120

\textbf{Job and/or Industry Outlook}: 22.24\% growth projected between 2006-2016 

\textbf{Other Information}:

\textbf{Questions to ask a worker in this field}:
\break What access technology would be required for work in this field? 
\break What are your hours usually like?
\break Is there a lot of variety in what you do from day to day?
\break Is your firm large or small?
\break How did you get your job?
\break What sort of preparation did you have? 
\break What are your goals for the future?
  
\pagebreak \section*{Module 3:  Finding a Job Mentor}\addcontentsline{toc}{section}{Module 3:  Finding a Job Mentor}\extramarks{Module 3}{COURSE 2: Career Exploration - Resources and Methods}
\noindent\makebox[\textwidth]{\rule{\linewidth}{0.4pt}} \etocsetnexttocdepth{4} \localtableofcontents 
\noindent\makebox[\textwidth]{\rule{\linewidth}{0.4pt}} 

\pagebreak \subsection*{3-1\quad  Lesson: Making contact with a mentor}
\addcontentsline{toc}{subsection}{3-1\quad  Lesson: Making contact with a mentor}
Most successful people will say that there have been important individuals who have influenced their career path, provided career advice, or offered support or an experienced perspective throughout their professional lives. These mentors are crucial members of any professional's support system. Some mentor relationships will develop naturally over the course of your career, but when you're job hunting for the first time it's a very good idea to actively seek out a mentor in your field of interest.
\subsubsection*{Contacting Mentors}
Below is a list of tips on contacting mentors, along with some questions that you may want to ask your mentor(s) once you've established a connection. It's important to limit yourself to a few thoughtfully chosen questions so that your mentor can spend their time on the topics that are most important to you.

Let the mentor know that you would like to learn more about the work he or she does. You may ask your own questions or use some of those listed below.

Thank the mentor for his or her time and for answering your questions.
You never know where you will make good professional contacts. Spell check any written messages prior to sending.

Whenever you interact with a mentor, be appropriate in your behavior and language and act in a manner that demonstrates you are serious about your job search.

If you are contacting a mentor via the phone, it's just as important to be appropriate, gracious, and polite. Act in a manner that shows you are serious about your job search.

\subsubsection*{Sample Questions}
Before contacting a mentor, you may want to write out a list of questions to ask. Earlier you started thinking about questions to ask someone in your job of interest. Review those questions and think about which are most important to you right now, at the beginning of your job analysis.

The following list may contain a question or two that you'd like to include at this stage. DO NOT copy and paste all of the questions below and send them to a mentor at once. Be thoughtful about what you really want to know about the mentor's job at this point and let the conversation evolve from there.

How did you find your job?

How long have you had your job?

Where did you receive your training for this job? 

What jobs did you have before this one?

Did you take vocational courses in high school, college or trade school that you recommend I consider? 

Did you participate in an internship or an apprenticeship?

Does your present company offer on-the-job training?

What is a typical starting salary for this job? What is your typical workday like?

How do you get to and from work? How do you perform your job duties?

Do you use specialized tools or equipment to perform your job duties?

How did you finance the purchase of any specialized equipment you use on the job?

Where and from whom did you receive training in how to operate the tools you use to perform your job? 

What related jobs do you know of that I might want to investigate?

What are your current career goals?

\subsubsection*{After You've Contacted a Mentor}
Some mentors will be quick to respond to your message or phone call. Others may take some time to get back to you, and some may not be able to respond to you at all. Be patient and remember that mentors are working professionals and often have limited time.

Don't Stop Now!

Fostering mentor relationships with people in all aspects of your life can be a great way to form a support system and build your network. There is no reason to limit yourself to only one, or one type of mentor. As your life progresses and your goals change, you may find new people to help guide and serve as role models for you. They may be in your field or outside of it, visually impaired or not, far more experienced than you, or only moderately so. Each of these relationships can prove valuable to you, either for providing different perspectives on the same subject, or for offering different types of advice altogether. The more varied your mentor relationships are, the more beneficial these relationships will be to you.

In addition to the benefits of advice and support, mentors have their own personal and professional networks. At various points along your career path, one or more of your mentors may be able to offer connections that could benefit your job search. We've discussed the importance of expanding your network; this is just another way to accomplish this task.

When you think about it, almost anyone can be a mentor: your parents, siblings, friends, teachers, colleagues, even acquaintances. Fostering a new mentor relationship can be easy-just ask for advice and guidance and be appreciative when you receive it. Most people are glad to share their experiences and support younger or less experienced colleagues or friends.

Get out there and get connected to other mentors!
 
\pagebreak \subsection*{3-2\quad  Assignment: Job Mentor}
\addcontentsline{toc}{subsection}{3-2\quad  Assignment: Job Mentor}
\subsubsection*{Contact Mentors}
Thank your mentor for their time and let them know you'd like to ask them a few questions. Ask a few thoughtful questions at the beginning of your interaction and see how things progress from there.

Make sure to set up a resource log record for each of your mentor contacts.

Expand on your Job Information Form with the information you learn in your interview. Fill in the gaps you were unable to locate through other means or start a new section for each mentor so you can record his or her background, training, education, past employment, and experiences with the job.
\pagebreak \subsection*{3-3\quad  Example Assignment: Laura’s Mentor Resource Log Record and Updated Job Information Form-After Mentor Interview}
\addcontentsline{toc}{subsection}{3-3\quad  Example Assignment: Laura’s Mentor Resource Log Record and Updated Job Information Form-After Mentor Interview}

\textbf{Contact Name}: Karen Barlow 

\textbf{Organization}: Barlow and Smythe 

\textbf{Job Title}: Lawyer

\textbf{Relationship to You}: Mentor 

\textbf{Phone Number}:

\textbf{Email Address}:

\textbf{Description}: Karen is a lawyer in Tallahassee. She is a partner in a law firm and was very responsive to my query for an interview.

\textbf{Job Information Form}: Updated

\textbf{Common Job Title(s)}: Paralegal, Legal Assistant, Legal Investigator, Patent Agent

\textbf{Major Job Duties}: Assist lawyers via research, document and case preparation. Conduct research to support a legal proceeding. Handle legal correspondence, maintain organization of documents in an established paper or electronic system. Prepare briefs, pleadings, appeals, wills, contracts, etc. Contact witnesses, meet with clients. Investigate facts and law of cases and conduct background research. Direct and coordinate law office activity. File pleadings. Gather and analyze research data, such as statutes, decisions, and legal articles.

\textbf{Required Education and/or Certification(s)}: Training in vocational school, related on the job experience, or an associate's degree. Usually need one or two years of training.

\textbf{Required Professional Experience}: One or two years of experience; might have a formal apprenticeship structure

\textbf{Other Qualifications (special computer skills, languages, etc.)}: Fluency with computers, hardware and software.

\textbf{Position this Job Reports to}: Lawyer, Partner in a Law Firm, Other Manager 

\textbf{Hours}: Full-time

\textbf{Location}: Everywhere Salary: \$46,120

\textbf{Job and/or Industry Outlook}: 22.24\% growth projected between 2015-2025

\textbf{Other Information}: Karen Barlow says that her firm is constantly seeking good paralegals and that they have a system where someone like me can get on the job training as long as I'm getting at least an associate's degree in the field.

\textbf{Questions to ask a worker in this field}: 
\break What access technology would be required for work in this field? 
\break Is there a lot of variety in what you do from day to day? 
\break Is your firm large or small?

\textbf{What are your hours usually like? }
\break Karen says that paralegals often work very long hours when the firm has a heavy case load. The paralegals at her firm are busy all the time and often come in early or stay late to make sure that everything is set for upcoming cases

\textbf{How did you get your job?}
\break Karen herself started as a paralegal, in order to get a feel for the law field. She liked it so much she decided to become a lawyer. She got her first job as a paralegal through her personal network-her dad knew a lawyer who was looking for a legal assistant at his firm. She was about to graduate from college with a degree in English literature and took the job to see if she wanted to apply to law school.

\textbf{What sort of preparation did you have?} 
\break Karen learned on the job. She had very good computer skills and was highly organized, two things that are important for paralegals. She said that she had to learn a lot about basic law procedure very quickly in order to get up to speed. She read a lot at home because often there wasn't enough time at work to do so. Her firm was accommodating of her visual impairment and didn't have a problem with her using access technology.
 
\pagebreak \section*{Module 4:	The Occupational Interview}\addcontentsline{toc}{section}{Module 4:	The Occupational Interview}\extramarks{Module 4}{COURSE 2: Career Exploration - Resources and Methods}
\noindent\makebox[\textwidth]{\rule{\linewidth}{0.4pt}} 
\etocsetnexttocdepth{4} 
\localtableofcontents 
\noindent\makebox[\textwidth]{\rule{\linewidth}{0.4pt}} 


\pagebreak \subsection*{4-1\quad  Lesson: Conducting an interview with a professional in your field of interest}
\addcontentsline{toc}{subsection}{4-1\quad  Lesson: Conducting an interview with a professional in your field of interest}
Occupational interviews are meetings set up to answer your questions about a field or position. These types of interviews are conducted with workers who are willing to take the time to speak with you and share their experiences.

This type of interview is not a job interview. Rather, it's understood that the sole purpose is information gathering, much like the way reporters use interviews to find background information to support a story. People who are willing to talk to you will most likely be enthusiastic about sharing their knowledge and experience, but you will be expected to be prepared with a clear sense of what you want to know about the job.

Occupational interviews can provide a wealth of information about the duties and responsibilities required to work in your field of choice. They also give you the opportunity to find out how a real-world business defines the role for the type of position you are investigating. During an occupational interview, you may find that your expectations for a job are different from the daily reality of the position.

It will be important to be persistent and to contact multiple organizations in order to find a company or employee willing to take the time to speak with you. Not all positions with similar titles will have the same range of duties at different organizations, or even within the same organization. Businesses continuously tailor their positions to meet changing business needs and/or to take advantage of individual employees' strengths, aptitudes, and interests.

\pagebreak \subsection*{4-2\quad  Setting up an Occupational Interview}
\addcontentsline{toc}{subsection}{4-2\quad  Setting up an Occupational Interview}
Below are some tips to help you with the process of setting up and successfully conducting an occupational interview.

\subsubsection*{Finding Contacts}

Think of the people in your network. Have you already made contact with individuals who might be able to help you connect with appropriate organizations or businesses? It's always easier to build on an existing contact than it is to start fresh with someone new. Ask around: you might be surprised by who has connections in your field of interest.

Use an Internet search engine to find businesses or organizations that have the type of job you're interested in learning about. Consider transportation to the location. Make sure that the businesses you choose to contact are accessible to you. Also consider the safety of the location.

Compile a list of the organizations you find that seem appropriate and accessible to you.

\textbf{Before You Call}

Think about what you want to say. Here are some questions to consider.
\begin{itemize}[leftmargin=*]
\item How will you describe the purpose of the interview?
\item What kind of experience you are looking for?
\item Do you want to talk with one person, or would it be helpful to talk with a few different people at the company?
\item How much time do you expect to need? (An informational interview is usually about an hour long. Remember that your interview will take time out of your interviewee's workday.)
\item Are there specific responsibilities that you would like to be able to see firsthand? For instance, if the position requires customer contact, ask if you can hold the interview during standard business hours so you can see how customer interactions are handled.
\end{itemize}
Keep in mind that not all organizations or workers will be open to an informational interview.

\textbf{When You Call}

Remember to be polite and professional with everyone you speak to.
Take notes on the places you contact and who you speak with. Make sure to update your resource log, because you might make contacts that you might want to be in touch with again in the future.

If a company does not allow informational interviews, be gracious and ask if they might know of other local companies in the same field that you could contact.

If a company is willing to set up an interview, ask if they can supply you with any background materials that might help you prepare. If you don't know about the individual to whom you will be speaking, ask for some basic information: job title, history with the company, etc.

\textbf{Before the Interview}

Schedule your transportation well in advance of the interview date, and make sure that you will arrive early. If the company requires that you fill out paperwork for confidentiality or other matters, ask if it can be emailed to you before your appointment.

Make sure you have appropriate attire for the workplace you will be visiting. If you're not sure about the dress code, ask!

\textbf{Preparing for and Conducting the Occupational Interview}

Thorough preparation is important for a successful occupational interview. Do diligent background research about the company. Pay careful attention to their website, familiarize yourself with the products, services, or activities that are central to the business.

Have a clear understanding of what you want to learn. An hour might sound like a long time, but it can go by very quickly.

Put together a list of questions well in advance of the interview and review, edit, and add to it regularly. Take a look at the questions you've been compiling on your Job Information Form, and the questions you've been discussing with your mentor(s). Can you build upon these sources for your occupational interview?

Make sure to ask about job duties that may not be typical to the position, or that are shared by coworkers in the office. Note the tasks mentioned and think about how you might accomplish them.

Your clothing should be appropriate for the workplace where the observation is being conducted. Good hygiene should be followed prior to the observation or any interaction with the employers.

\textbf{Important things to remember}
\begin{itemize}[leftmargin=*]
\item Be early or on time for your appointment
\item Be positive
\item Be polite and gracious
\item Act in a professional manner
\item Use appropriate language
\item If allowed, take notes. If not, pay attention and take notes once the interview is over
\item If you are using technology, be professional about it
\item Make sure you are not a distraction
\item Be prepared to answer questions
\end{itemize}
\textbf{After the Interview}

Keep track of the contact information for the person you spoke to. It's very important to send a message thanking the interviewee for taking the time to speak to you.

If you say that you will keep in contact and update the person on your job search, then keep in touch! You never know where job leads will come from, and this person could become part of your personal network. You could send a message to the organization, mentioning how great the person was for allowing you to do an occupational interview and how helpful the experience was for you. Employers always like to know positive information and hear compliments about their employees.

\pagebreak \subsection*{4-3\quad  Assignment: Occupational Interviews}
\addcontentsline{toc}{subsection}{4-3\quad  Assignment: Occupational Interviews}
Find and conduct an occupational interview with a professional working in the job you are analyzing. Update your resource log with all of the contacts you make during this process and continue to flesh out your Job Information Form.

\pagebreak \subsection*{4-4\quad  Example Assignment: Laura Smith’s Mentor Resource Log Record and Updated Job Information Form-After Mentor and Occupational Interview}
\addcontentsline{toc}{subsection}{4-4\quad  Example Assignment: Laura Smith’s Mentor Resource Log Record and Updated Job Information Form-After Mentor and Occupational Interview}

\textbf{Contact name}: Lindsey Chapin 

\textbf{Organization}: Barlow and Smythe 

\textbf{Job Title}: Paralegal

\textbf{Relationship to You}: paralegal who agreed to do an occupational interview Phone Number: Email Address:

\textbf{Description}: Lindsey is my first occupational interview subject. 

\textbf{Job Information Form}: Updated

\textbf{Common Job Title(s)}: Paralegal, Legal Assistant, Legal Investigator, Patent Agent

\textbf{Major Job Duties}: Assist lawyers via research, document and case preparation. Conduct research to support a legal proceeding. Handle legal correspondence, maintain organization of documents in an established paper or electronic system. Prepare briefs, pleadings, appeals, wills, contracts, etc. Contact witnesses, meet with clients. Investigate facts and law of cases and conduct background research. Direct and coordinate law office activity. File pleadings. Gather and analyze research data, such as statutes, decisions, and legal articles. Ensure law library is up to date.

\textbf{Required Education and/or Certification(s)}: Training in vocational school, related on the job experience, or an associate's degree. Usually need one or two years of training.

\textbf{Required Professional Experience}: One or two years of experience; might have a formal apprenticeship structure

\textbf{Other Qualifications (special computer skills, languages, etc.)}: Fluency with computers, hardware and software.

\textbf{Position this Job Reports to}: Lawyer, Partner in a Law Firm, Other Manager Hours: Full-time

\textbf{Location}: Everywhere Salary: \$46,120

\textbf{Job and/or Industry Outlook}: 22.24\% growth projected for years 2015-2025

\textbf{Other Information}: My mentor Karen Barlow says that her firm is constantly seeking good paralegals and that they have a system where someone like me can get on the job training as long as I'm getting at least an associate's degree in the field.

\textbf{Questions to ask a worker in this field}:

\textbf{What access technology would be required for work in this field?} 

In the occupational interview, Lindsey spent some time describing to me the types of work that she has to do. She asked me questions about what sort of access technology I use and then we both talked about what would probably be required to be a paralegal. It seems pretty straightforward-I'd need a screen reader and OCR software and maybe some assistance with the paper filing system.

\textbf{What are your hours usually like?} 

Karen says that paralegals often work very long hours when the firm has a heavy case load. The paralegals at her firm are busy all the time and often come in early or stay late to make sure that everything is set for upcoming cases. Lindsey said that the hours are often really long but that it's fun if you like the feeling of working hard to meet a deadline.

\textbf{Is there a lot of variety in what you do from day to day? }

Lindsey said that while there is a lot of variety in the work she does, often there are several days in a row that are consumed with filing, paperwork, and taking care of office management tasks. She said that every now and again it can get boring or repetitive, but that is more than made up for by the fact that you get to see some pretty interesting cases.

\textbf{Is your firm large or small?} 

Small firm. Lindsey talked about how she chose a small firm specifically because she wanted to feel like she was part of a team and not just a cog in a huge organization. The trade-off is that she doesn't make as high a salary as some of her friends at the larger firms.

\textbf{How did you get your job?} 

Karen herself started as a paralegal, in order to get a feel for the law field. She liked it so much she decided to become a lawyer. She got her first job as a paralegal through her personal network. Her dad knew a lawyer who was looking for a legal assistant at his firm. She was about to graduate from college with a degree in English literature and took the job to see if she wanted to apply to law school.

Lindsey found the job listing at her career center and after sending out some feelers to her network found out that one of her professors knew Karen Barlow. The professor put in a good word for Lindsey after she applied for the job. She said the interview was pretty tough, but everyone was nice.

\textbf{What sort of preparation did you have?} 

Karen learned on the job. She had very good computer skills and was highly organized, two things that are important for paralegals. She said that she had to learn a lot about basic law procedure very quickly in order to get up to speed. She read a lot at home because often there wasn't enough time at work to do so. Her firm was accommodating of her visual impairment and didn't have a problem with her using access technology.

Lindsey had spent a summer as a paralegal between her junior and senior year, so she had some training already. She's always wanted to be in the law field, so she majored in pre-law in college, so she had some good preparation and familiarity with a lot of basic law.

What are your goals for the future? Lindsey is applying to law school. She's going to become a lawyer and hopes to be a partner in her own firm one day.

Are there any special skills you need to do this work? Lindsey said that it would be best to have a good understanding of how Lexis/Nexus, an online law literature database, works. She said that she uses it all the time and you have to be fast and efficient when you're looking things up. She also said that since the lawyer she works for specializes in labor law, she needs to be able to support that work with good knowledge about the field as well.
 
\pagebreak \section*{Module 5:	Job Observation}\addcontentsline{toc}{section}{Module 5:	Job Observation}\extramarks{Module 5}{COURSE 2: Career Exploration - Resources and Methods}
\noindent\makebox[\textwidth]{\rule{\linewidth}{0.4pt}} \etocsetnexttocdepth{4} \localtableofcontents 
\noindent\makebox[\textwidth]{\rule{\linewidth}{0.4pt}} 


\pagebreak \subsection*{5-1\quad Lesson: Conducting a job Observation}
\addcontentsline{toc}{subsection}{5-1\quad Lesson: Conducting a job Observation}
There are few experiences that will be more valuable to you than an opportunity to observe your job of interest being performed in the real working world. Aside from actually doing a job yourself, a job observation is one of the best ways to learn about the realities of any position.

Job observations require much more of a time commitment from both the employee who will be observed and the company for which he or she works. It might be challenging to find a company willing to offer this opportunity. Be persistent, use your network. Ask your mentor for contacts or ideas about places you might query.

During your observation, it's important not to judge, criticize, or comment on what you observe. Your goal as an observer is not to assess how the work is being performed, but to learn as much as you can about what the job actually entails on a day-to-day basis. Look at the duties and responsibilities that are required and think about whether or not the reality of the position is appealing to you.

Remember to express your gratitude and appreciation for the opportunity to observe. A thank you note is appropriate after this type of experience.

\pagebreak \subsection*{5-2\quad Assignment: Job Observation}
\addcontentsline{toc}{subsection}{5-2\quad Assignment: Job Observation}
Locate and set up an appropriate a job observation. Support your search for an observation with good research and requests for help to your network.

The points and questions provided below are offered as guidance for your observation. What else are you interested in learning? Make sure to take good notes, either while you are observing or immediately after the observation is finished or you're in a place where you can concentrate.
\begin{itemize}[leftmargin=*]
\item Job tasks that are performed
\item Methods used to accomplish job tasks
\item Technology utilized in job tasks
\item Specific computer applications utilized to accomplish the job
\item Where is the job performed?
\item Physical location such as lighting, space, desk set-up, number of people in the office
\item What is the atmosphere or feeling in the work environment?
\item Does the employee interact with coworkers?
\item Does the employee interact with customers?
\item Does the employee use the phone to perform their duties?
\item Is the phone accessible?
\item What percentage of the day is spent interacting with others?
\item Does the person stand for long periods of the day?
 \item Is the employee required to lift things? If so, are these objects manageable for you?
\item What hours and days does the employee work?
\item Is this a typical day on the job?
\item Are there tasks that you would find difficult to perform? Which? How does the employee perform the tasks?
\item How quickly are tasks performed? Is speed in performing job tasks a requirement? Will speed in performing tasks be a requirement when applying for this type of job?
\item What skills or training would you need to successfully fulfill the duties? 
\item Is the training something that you expected to be necessary? Could you get this training?
\item Could you learn the skills you see being used in the job?
\item Is the supervisor on site?
\item How many employees work in the location?
\end{itemize}
For your assignment, take some time to think about how closely this observation paralleled your ideas about the job. Were there any surprises? What are you excited about? Any disappointments? Do you still think that this position is a good match for your skills, personality and values? Are you still interested in this type of work? Explain why or why not.

\pagebreak \subsection*{5-3\quad Example Assignment: Laura Smith's Job Observation Report and Reflection}
\addcontentsline{toc}{subsection}{5-3\quad Example Assignment: Laura Smith's Job Observation Report and Reflection}
I wanted to do an observation at a law firm larger than Karen Barlow's, because that's the sort of environment I thought I'd ultimately want to work in. Karen helped me set up an observation at Sato, Lomanaco, and Partners, where I was able to observe one of their paralegals, Simon Craig.

Simon was really patient and helpful and took a lot of time out of his day to explain everything to me, which wasn't something I was expecting. It was a totally different feel than the Barlow office. The firm is large enough. There are 8 partners, 2 junior partners and 7 staff attorneys-that the paralegals are all working in one section of the office. None of the paralegals had a real office, they were all at open workstations and cubicles. There didn't seem to be a lot of face time with the lawyers. Communication was either by phone or email.

I was really surprised by how much paperwork Simon had to do. His desk was piled with stacks of paper, and he said that I was seeing him on a good day! Just in the time I was there today, file boxes for three new cases were dropped off at his desk. He spent a lot of time filing, copying, and documenting various procedural issues for about seven cases. He said he really liked the job but found it very stressful at times because each attorney has a different way of doing things and it was hard to keep everyone straight sometimes. I had been thinking that each lawyer would have an assigned paralegal, but apparently the way this firm does it is that most paralegals work for all of the junior partners and staff attorneys.

The partners each have an assigned paralegal, but they also can pull from the paralegal ``pool'' any time they need extra support. There were about 20 paralegals, and everyone seemed really, really busy. The environment felt a little tense at times, and I overheard one of the paralegals talking to someone who seemed really angry about a mistake the paralegal had made. It also seemed pretty exciting, though. Simon said that he liked the cases he worked on-they were complex and interesting, and he'd learned a lot. His plan is to go on to be a lawyer, and he said that he felt his job was really preparing him well for the future.

I liked the energy in the office and the got the sense that everyone was working really hard and being productive. I observed Simon issue a subpoena and also make several calls to witnesses who will be needed for upcoming trials. It was all very interesting and varied work.

I am still interested in becoming a paralegal, though I think I might aim for getting hired at a mid-range firm. Sato and Partners seemed a little too stressful as a place to start. Maybe after I have a couple years under my belt in a smaller environment, I'll either move up to a larger firm or go get my law degree.

The work is definitely a good match for my skills, personality, and values. I like that there is a good degree of interaction with a variety of people and I also like how much organization is required. I think if I can find the right firm, I'll really enjoy working as a paralegal.
 
 
\pagebreak \section*{Module 6:	Discrepancy  Analysis}\addcontentsline{toc}{section}{Module 6:	Discrepancy  Analysis}\extramarks{Module 1}{COURSE 6: Career Exploration - Resources and Methods}
\noindent\makebox[\textwidth]{\rule{\linewidth}{0.4pt}} \etocsetnexttocdepth{4} \localtableofcontents 
\noindent\makebox[\textwidth]{\rule{\linewidth}{0.4pt}} 


\pagebreak \subsection*{6-1\quad Lesson: Determine whether you are moving in the right direction}
\addcontentsline{toc}{subsection}{6-1\quad Lesson: Determine whether you are moving in the right direction}

\textbf{Finding the Right Fit}

Most people work in jobs or pursue careers where their skills, abilities, interests, values, work personality, education, and professional experience are a good fit for the duties they must perform on a day-to-day basis. In order to be hired, applicants for a specific position must already possess most, if not all, of the experience, skills, and education, required to perform the job.

In your assignment for this section, you will compare your self-assessment from Module 1 to the job analysis you've performed for your job of interest in this module. The purpose of this comparison is to determine how well the skills and traits you currently possess match those required for your job of interest. You will consider a number of things like your work personality, your education, your strengths and weaknesses, as well as your interests, abilities, and values. You may want to refresh your memory about the assignments you completed in Module 2, as this information will play an important role in this process.

\textbf{What is a Discrepancy?}
In the context of this lesson, a discrepancy is the difference between the skill that you possess and the skill that is necessary to be successful in a specific career or job. For example, if you have a typing speed of 40 words per minute and a job you're interested in requires a speed of 130 words per minute, there is a discrepancy between your typing speed and the speed required by the job.

\textbf{Why are Discrepancies Important?}

The purpose of identifying the discrepancies between your skills and those required by your job of interest is to help you intelligently decide your next step. Analyzing discrepancies will help you answer important questions. Is it possible to perform your job of interest given your current skills? Do you meet the minimum skill levels required, or do you need more training before you apply? Are there aspects of the job that, now that you know more about it, make it less appealing to you? Does the job take advantage of your strengths and interests? 

\textbf{Are you overqualified for the position?}

Discrepancy analysis allows you to take an objective look at what you might need to work on to make a given job or career a good choice for you. In order to be a successful job seeker, you must habitually compare your skills and aptitudes to those required by every job you consider.
Employers also always use discrepancy analysis when looking for employees. Many typically create a checklist of required skills and aptitudes based on the job requirements and will evaluate every applicant against this criterion.

Often employers have basic criteria that determine whether they disregard an application right away. Education, years of experience, a familiarity with specific software, or mastery of a specific type of machinery can be very crucial to success in a given job and it may not be reasonable for the employer to consider applicants who don't meet those criteria.

It's strongly suggested that you practice discrepancy analysis and utilize the information you have collected in prior lessons. A good way to get a lot of practice beyond that offered by your own job analysis is to take job listings from job websites, break each listing down into a checklist of requirements, and then compare this list to your self-analysis. You might find a great fit that you'd not thought of before!
 
\pagebreak \subsection*{6-2\quad Assignment: Discrepancy Analysis}
\addcontentsline{toc}{subsection}{6-2\quad Assignment: Discrepancy Analysis}
Go back to your Self-Analysis Report and review your interests, skills/abilities, values, and work personality code. Take a moment to review your Job Information Form, which now should be quite fleshed out and full of the relevant information you've researched through job descriptions, the APH CareerConnect or NSITE Connect websites, and your occupational interview and job observation. Now write a detailed discrepancy analysis, taking into account the areas you think you are prepared for and the areas you need to work on in order to be a viable candidate for this position. The goal is to identify the places on which you should focus your attention to make this career a good fit for you.

\pagebreak \subsection*{6-3\quad Example Assignment: Laura Smith's Discrepancy Analysis}
\addcontentsline{toc}{subsection}{6-3\quad Example Assignment: Laura Smith's Discrepancy  Analysis}
\textbf{Job}: Paralegal

\textbf{Analysis}: On the whole, I think I'm pretty well prepared for this job and the tasks and nature of the work seem like a good fit for my skills and traits.

\textbf{I have discrepancies in the following areas}: Legal database experience: I don't have any experience with Lexis/Nexus or any other of the main legal databases. I think I need to get up to speed on how to use these resources so that I can bring that experience and confidence to my applications and interviews.

\textbf{Area of specialty}: I was thinking that I could become a paralegal and then figure out what sort of lawyer I want to be, or that I would just get a good body of knowledge from whatever law the firm I got a job at specialized in. It seems to me now that I should probably figure out what sort of law holds the most interest to me and then direct my search based on that. That way my paralegal work can inform my ultimate goal of becoming a lawyer. This will require some research on my part before moving ahead with finding a job.

\textbf{Paper filing systems}: I'm a little concerned about how to manage complex paper filing systems that haven't been made accessible for people with visual impairments. Paralegals have to deal with a lot of paper really quickly, so I don't want this to be an issue for a potential employer. I'd like to have a solution ready if it comes up.

\textbf{Education/experience}: I'm going to be finishing up my BA this year, so I think I'm prepared, though I wish I'd thought to try to get some paralegal experience last summer. I wonder if there is a way for me to start part time somewhere during my last year.

\textbf{Typing}: while no one has overtly said there is a requirement for typing speed, so much communication happens via email in a law office that I think it might be a good idea for me to get my typing speed up just so I can save some time during the day. I don't think it will make much difference when applying for a job, but it might when it comes to keeping the job I finally land.
 
 
\pagebreak \section*{Module 7:	The Vocational Action Plan Assignment}\addcontentsline{toc}{section}{Module 7:	The Vocational Action Plan Assignment}\extramarks{Module 7}{COURSE 2: Career Exploration - Resources and Methods}
\noindent\makebox[\textwidth]{\rule{\linewidth}{0.4pt}} 
\etocsetnexttocdepth{4} 
\localtableofcontents 
\noindent\makebox[\textwidth]{\rule{\linewidth}{0.4pt}} 


\pagebreak \subsection*{7-1\quad Lesson: Putting your research to work for you}
\addcontentsline{toc}{subsection}{7-1\quad Lesson: Putting your research to work for you}
In this module, you explored a job you're interested in and in your discrepancy analysis you compared the requirements of that position to your skills, abilities, and liabilities. Hopefully the job you've analyzed has proven to be a good fit for you. If not, you should move to a new job and start the analysis process over again until you find a position that feels right. Once you've found the job you'd like to pursue, your next step is to create a vocational action plan.

Creating a vocational action plan is a big step towards achieving your ultimate goal of employment. The action plan will help you maintain self-awareness, be realistic about goal setting, and break down your required progress into achievable steps.

\pagebreak \subsection*{7-2\quad Assignment: Vocational Action Plan}
\addcontentsline{toc}{subsection}{7-2\quad Assignment: Vocational Action Plan}
List the job that you are most interested in getting. This is your ultimate vocational goal. Based on your discrepancy analysis, think about the steps that will be necessary to reach that goal. Use the information from the discrepancy analysis to identify the requirements that you do not meet. Think about what is necessary to meet them-is it improving on an existing skill? Learning a new one? Getting certification? Finishing a degree? Now put together a step-by-step plan that you can implement in order to meet your goal.

\pagebreak \subsection*{7-3\quad Example Assignment: Laura Smith's Vocational Action Plan}
\addcontentsline{toc}{subsection}{7-3\quad Example Assignment: Laura Smith's Vocational Action Plan}
\textbf{Job}: Paralegal Plan

\textbf{Discrepancy}: Legal Database Experience

\textbf{Action}: See if I can access Lexis/Nexus through my university or branch library. See if there are librarians at these locations who could help me practice using these databases. Spend two hours per week working on Lexis/Nexus searches and gaining familiarity with how the database works. Talk to Karen, Simon, and Lindsey to see if there are any other common legal databases-or other online resources-I should familiarize myself with to make me a stronger candidate

\textbf{Discrepancy}: I haven't figured out what kind of law I'm interested in

\textbf{Action}: Talk to Karen about how she went about choosing her specialty. Before I call her, I'm going to do some research into types of law and see if I can narrow the field down a bit. I know I don't want to go the corporate route or be a bankruptcy lawyer.

\textbf{Discrepancy}: I don't have any work experience in the field

\textbf{Action}: I'm going to go to the career center to see if they have any leads on part-time paralegal, or related, work. Hopefully I can get some kind of real-world experience before I graduate.
 
\textbf{Concerns}:

\textbf{Typing}: I want to get my typing speed up a bit. I'm going to ask at the vocational rehabilitation center if they have any courses I can take to improve.

\textbf{Paper filing systems}: I want to talk to Lindsey about how realistic it is for a law firm to make their paper filing systems accessible to me. She seems to manage fine, but Karen's firm is really flexible and invested in making things accessible. If I'm going to be at a larger firm, I need to know how it might work so I can come up with solutions.

