\hypertarget{course1workbook}{}\chapter*{COURSE 1: Self-Awareness Workbook}\label{course1workbook}
\addcontentsline{toc}{chapter}{COURSE 1: Self-Awareness Workbook}
\extramarks{The JobSeeker's Toolkit}{COURSE 1: Self-Awareness Workbook}
\noindent\makebox[\textwidth]{\rule{\linewidth}{0.4pt}}
\localtableofcontents

%\noindent\makebox[\textwidth]{\rule{\linewidth}{0.4pt}} 

\pagebreak \section*{1-2 Assignment: Your Network Pyramid}\label{course1workbook1_2}
\addcontentsline{toc}{section}{1-2 Assignment: Your Network Pyramid} \extramarks{Module 1}{1-2 Assignment: Your Network Pyramid}
Write out who is included in your network pyramid, starting with you at the top level (level one) and filling in the names of your intimates (level two), friends (level three), acquaintances (level four), and paid helpers or casual acquaintances (level five).

\textbf{Level 1:} \hrulefill

\textbf{Level 2:} \hrulefill

\hrulefill

\hrulefill

\hrulefill

\hrulefill

\textbf{Level 3:} \hrulefill

\hrulefill

\hrulefill

\hrulefill

\hrulefill

\textbf{Level 4:} \hrulefill

\hrulefill

\hrulefill

\hrulefill

\hrulefill

\hrulefill


\textbf{Level 5:} \hrulefill

\hrulefill

\hrulefill

\hrulefill

\hrulefill

\hrulefill

\hrulefill
\clearpage
Looking at 1-3 on Page XX of the JobSeeker's Toolkit, please describe the positives and negatives of Denise's Network Pyramid. 

\hrulefill

\hrulefill

\hrulefill

\hrulefill

\hrulefill

\hrulefill

Which Levels do you think would provide Denise good job leads? 

\hrulefill

\hrulefill

In which level would Denise find a possible professional mentor? 

\hrulefill

\hrulefill

Why do you believe this?

\hrulefill

\hrulefill

\hrulefill

\hrulefill

Using what you have learned scrutinizing Denise's Network Pyramid, The next step is that you are going to look back to your network pyramid and use the form on the following page to answer the same questions about your own network pyramid. Please do not change any members of your pyramid, but you are free to place a mark next to those you think may be more helpful professionally than others. 
\clearpage
Please describe the positives and negatives of Your own Network Pyramid. 

\hrulefill

\hrulefill

\hrulefill

\hrulefill

\hrulefill

\hrulefill

Which Levels do you think would provide you good job leads? Whom?

\hrulefill

\hrulefill

\hrulefill

\hrulefill

In which level would you find a possible professional mentor?  Whom (may be more than one)

\hrulefill

\hrulefill

\hrulefill

\hrulefill

Why do you believe this?

\hrulefill

\hrulefill

\hrulefill

\hrulefill

\hrulefill

\hrulefill

\clearpage  \section*{2-2 Assignment: Your Network Expansion plan}
\addcontentsline{toc}{section}{2-2 Assignment: Your Network Expansion Plan} \extramarks{Module 1}{2-2 Assignment: Your Network Expansion plan}
In a paragraph or two, outline your strategy for expanding your professional network. Will you be joining any organizations or groups? If so, which ones, and how do you anticipate that joining will enhance your career growth? Additionally, consider how the information in this paragraph can guide your criteria for selecting a mentor. What qualities or expertise would you seek in a mentor to support your professional development?

\clearpage \section*{3-2 Assignment: Feedback}
\addcontentsline{toc}{section}{3-2 Assignment: Feedback} \extramarks{Module 1}{3-2 Assignment: Feedback}

During the next two days ask five different people (family members, friends, classmates, teachers, former co- workers, or others) for feedback about you. Ask each participant to provide 5 to 10 comments that describe you, balanced between strengths and weaknesses: the things they think you do well and those that could use improvement. Collect responses digitally; you'll be using them later. Be sure to include the name of the respondent and their relationship to you.
\subsubsection*{Part 2: Analysis}

Compile a list of the strengths and weaknesses you heard in your feedback. Describe what you liked hearing and what you did not like hearing. When listing areas for improvement, think about how you might go about making changes.

\clearpage \section*{4-2 Assignment: Interests}
\addcontentsline{toc}{section}{4-2 Assignment: Interests} \extramarks{Module 1}{4-2 Assignment: Interests}
Make a list of your top ten interests. What are the things you enjoy doing at home, at school, and in the community? List your interests as they occur to you and then go back and specify what it is that you like about each interest area.

Next, take some time to brainstorm about the skills you use when you're pursuing your interests, and also what professions might use these skills. Write down all of your ideas.

\clearpage \section*{5-2 Assignment: Skills and Abilities}
\addcontentsline{toc}{section}{5-2 Assignment: Skills and Abilities} \extramarks{Module 1}{5-2 Assignment: Skills and Abilities}
Make a list of ten skills you currently have. Rank the list of skills, with one being the skill with which you think you have the highest competence and ten, the skill with which you have the lowest.

Once you have your list, write down the jobs you think use these skills. Notice if there is overlap with the jobs you brainstormed in your interest inventory. Are your strongest skills related to the jobs mentioned in your interest inventory? If there isn't any overlap, or if your weaker skills are those related to jobs in your interest area, write down the ways you might bridge the gap between your interest-related jobs and your current skill set.

\clearpage \section*{6-2 Assignment: Values}
\addcontentsline{toc}{section}{6-2 Assignment: Values} \extramarks{Module 1}{6-2 Assignment: Values}
\begin{enumerate}[leftmargin=1cm]
	\item Make a list of ten of your values. Rank them from one to ten, with one being the most important and ten being the least important.
	      Explain how one of your values may have influenced a recent decision.
	\item Choose a person in your life whose values are important for you to take into consideration when deciding on a new job. Describe how this person's values influence your decisions.
	\item Next, choose a person in your life whose values differ from your own. Explain how that person's values differ from yours, and what your relationship is with that person.
\end{enumerate}

\clearpage \section*{7-2 Assignment: Work Personality}
\addcontentsline{toc}{section}{7-2 Assignment: Work Personality} \extramarks{Module 1}{7-2 Assignment: Work Personality}
\subsubsection*{Part 1: The Cafeteria Experiment}
Now that you have sense of how Holland has defined his work personality types, here is an activity that will help you determine which of these types are relevant to you. The following is an activity designed to help define your work personalities and was adapted from ``The Party Exercise'' from What Color is Your Parachute by Richard Bolles.

\subsubsection*{The Experiment}

Make an empty list with three spaces numbered 1 through 3.

Imagine that you are in a cafeteria, where different groups of people are sitting at six separate tables. Each table is labeled with one of the personality types from this chapter: R for realistic, I for investigative, A for artistic, S for social, E for enterprising and C for conventional. The people at each table have personalities dominated by the labeled type.

You have to choose which table to sit at. You can't sit in between the tables or in the middle of the cafeteria. Review the personality definitions above and think about which personality type you would most like to sit with. Record this letter next to the number one on your list.

The people you are sitting with at the table all decide they are leaving the cafeteria after fifteen minutes and the table will be folded up and put away. You must choose to join a second table. What would be your second choice now that your first one has been eliminated? List a second choice next to number two on the list you have created.

After fifteen minutes, the people you are sitting with at the second table decide to get up and leave as well. This table will be folded up and put away. You will now have to choose a third table to sit at. Think about the work personality types available, and then think which would be your next choice? Remember, the prior two are no longer available. What table would you choose? List the third choice next to the number three on your list.

Using Mike's list of work personalities as an example, his work code is:

2. S

3. I

4. A

His work code is SIA and it shows that Mike's dominant work personality type is Social.

Now create your own work personality code, which would be your top three choices listed in ranked order. Make a note of your code because it will be referenced throughout the remainder of this process.
\subsubsection*{Part 2: Work Personality Job Analysis}
For this assignment you should think back to your interests, skills and values to see how they relate with your work personality type. Use the work you have done in the other sections to help create a list of five jobs you believe would suit you. Consider all of the areas we have covered up to this point. You should put your work personality type and work personality code at the top, and then list the five jobs.

\clearpage \section*{8-2 Assignment: Your Portfolio System}
\addcontentsline{toc}{section}{8-2 Assignment: Your Portfolio System} \extramarks{Module 1}{8-2 Assignment: Your Portfolio System}
Create and organize your digital portfolio. The goal is to develop a system that will make it easy and fast to find your materials whenever you need to.
\begin{enumerate}[leftmargin=1cm]
	\item Identify the categories you will use to keep your files organized and provide a few examples of what will be contained in each category.
	\item Explain how you will name the folders and categories.
\end{enumerate}

\clearpage \section*{9-2 Assignment: Self-Analysis}
\addcontentsline{toc}{section}{9-2 Assignment: Self-Analysis} \extramarks{Module 1}{9-2 Assignment: Self-Analysis}
Using your prior work from this Course, create a comprehensive self-analysis in the following areas.
\begin{enumerate}[leftmargin=1cm]
	\item Personal network
	\item Feedback from others
	\item Interests
	\item Skills/abilities
	\item Values
	\item Work personality
\end{enumerate}